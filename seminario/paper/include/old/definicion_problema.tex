\section{Definición del problema}
Para entender el concepto de Electronic Market Making es necesario definir algunos términos: \cite{glosten1985bid}
\begin{itemize}
	\item Dealers: compran y venden "valores". Buscan comprar bajo y vender alto.
	\item Se fijan dos precios: \textbf{Bid} (dealer compra, cliente vende). \textbf{Ask} (dealer vende, cliente compra). Lo esperado es que: 
		\textbf{ask} > \textbf{bid}. El término \textbf{spread} corresponde a \textbf{ask} - \textbf{bid}, lo que corresponde a la diferencia entre lo que
		el dealer vende y compra (concepto de ganancia). Esto se entiende como la fuente de ingresos de los dealers. \cite{ho1981optimal}
\end{itemize}
En la literatura se han visto algunos resultados experimentales de este tema usando sistemas computacionales que proveen interfaces para desarrollar en sus sistemas, 
nuevamente para ello es necesario ir definiendo algunos conceptos entre medio:
\begin{itemize}
	\item Pen exchange simulator (PXS): usa Island Electronic communitacion network basado en limit orders.
	\item New York Stock Exchange (NYSE): operador financiero de mercado que provee tecnologías de trading innovadoras, además de data real
		de indicadores económicos y valores de activos. Organización creada desde el 2007.
	\item NASDAQ: este sistema es igual que el anterior y cae dentro de la categoría de \textbf{security dealers}. \cite{barclay1999effects}
	\item Market order: es una instrucción del cliente al dealer (comprar o vender al mejor precio posible). Asegura la realización de la transacción, 
		pero no el precio.
	\item Limit order: es una instrucción del cliente al dealer de pedir una transacción a un precio específico (o más ventajoso). 
		No garantiza la realización de la transacción pero si se conoce una cota del precio.
	\item Existe un order book, el cua es un registro de dos colas: una de venta y una de compra. Si una nueva orden entra y no es satisfecha por el limit order, 
		esta ingresa al order book (cola de venta o compra).
	\item PXS a cada iteración hace:
	\begin{itemize}
		\item Toma un snapshot del island order book.
		\item Reúne toda las limit orders desde los agentes en simulación.
		\item Combina todas las órdenes (reales y ficticias) usando las reglas del ECN. Algunas transacciones se realizan y otras entran al island order book.
		\item Cuando se realiza una transacción, se actualizan los agentes (stock y \$) y el order book.
	\end{itemize}
\end{itemize}
El electronic market maker se descompone en dos áreas grandes de investigación:
\begin{itemize}
	\item Etablecer el bid-ask spread
	\item Actualizar el bid-ask spread, de manera predictiva o no predictiva.
\end{itemize}
La primera decisión para el market maker es donde establecer el spread inicial. Valoración de la seguridad al ser tradeado:
\begin{itemize}
	\item Para un stock, determinar el valor de este para la compañía
	\item Para un bono buscar el valor presente del pago prometido
	\item Si no hay mercado establecido, o el mercado tiene poca liquidez, entonces la valorización es la única aproximación. \cite{seppi1997liquidity}
\end{itemize}
La segunda, actualizar el bid-ask, es considerado como el corazón del market maker. Existen formas predictivas al corto plazo teniendo en cuenta el comportamiento 
de la curva (bid-ask), esto es posible cuando se tienen valores históricos, y en base a ellos se pueden calcular aproximaciones a los nuevos valores posibles. 
El segundo caso posible, es que no se posea información histórica, y que se tenga la información actual del mercado. Estas últimas son inherentemente más simples.

En el estudio de la actualización del bid-ask mediante formas predictivas es donde se utilizan la mayor parte de procedimientos matemáticos. Además en el presente
se trabaja con datos de alta frecuencia, es decir, data proveniente de los servidores con diferencia temporal muy pequeña, lo que deja abierta la posibilidad
de realizar investigación tanto matemática como computacional en la optimización de estos cálculos.
