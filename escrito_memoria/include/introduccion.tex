\parindent=1.5em
Esta memoria está enfocada a abordar un problema financiero, relacionado con las series financieras de alta frecuencia y una forma particular
de poder realizar pronósticos. Se pretende abordar esta problemática con metodologías computacionales, aplicando algunos conceptos matemáticos
útiles para esta área.

La naturaleza del problema es de carácter financiero, por lo que es necesario introducir algunos conceptos, criterios y términos generales
asociados al área.

Este capítulo tiene como objetivo introducir los conceptos de series de tiempo financierras, sus orígenes y la alta frecuencia.

\section{Mercados Financieros}

El mercado financiero es un espacio con marco institucional que permite poner en contacto a oferentes y demandantes para que efectúen
transacciones financieras. La idea de mercado como foro organizado a la que acuden agentes económicos para efectuar transacciones
queda reducida en el mundo financiero como las bolsas de valores.

El concepto de mercado financiero se utiliza en general para referirse a cualquier mercado organizado en el que se negocien instrumentos financieros
de todo tipo, como por ejemplo, acciones. Además el espacio para generar estas interacciones no necesariamente debe ser físico. Por otro lado el negociar
instrumentos financieros implica a grandes rasgos: intercambiar instrumentos financieros, y definir su precio. Por ende, estos mercados están basados en
las fuerzas de oferta y demanda, ubicando a todos los oferentes en el mismo lugar para facilitarle la búsqueda a los demandantes.

Una de las razones que hace importante este tipo de mercado, es su funcionalidad, ya que permiten:
\begin{itemize}
        \item Aumentar el capital, siendo esto uno de los casos favorables, ya que también hay probabilidades considerables de disminuir el capital.
        \item Comercio internacional, como en los mercados de divisas, por ejemplo Forex.
        \item Reunir a quienes necesitan recursos financieros, con los que tienen recursos financieros.
\end{itemize}

En este tipo de mercado se definen los siguientes conceptos:
\begin{itemize}
	\item \emph{Dealer}: Un dealer es un ente, presente en los mercados que están dispuestos a comprar o vender.
	\item \emph{Orders}: Operación de compra/venta de activos.
	\item \emph{Bid price}: Precio al cual un \emph{dealer} está dispuesto a comprar.
	\item \emph{Ask price}: Precio al cual un \emph{dealer} está dispuesto a vender.
	\item \emph{Market orders}: instrucción del cliente al dealer, comprar o vender al mejor precio posible dentro de los valores actuales del mercado.
		Esto asegura la realización de la transacción, pero no el precio.
	\item \emph{Limit orders}: es una orden para comprar a un valor máximo (precio determinado), o para vender a un valor mínimo (precio determinado).
		Esto le da al cliente el control sobre el precio al que se ejecuta el comercio, sin embargo, no garantiza la realización de la transacción.
\end{itemize}

El conjunto de \emph{Limit orders} forman los \emph{books} para cierto activo, los cuales proveen información detallada de dicho instrumento. Con estos datos
se forman los llamados bid-ask spreads, que es la diferencia entre el precios cotizados para una venta inmediata (oferta) y una compra inmediata (bid). 
Además se generan los bid-ask qoute, el cual define cotas para el precio de transacción.

\subsection{Mercados bursátiles}
Los mercados bursátiles están clasificados en los mercados de capitales, en donde se negocian activos financieros. Este tipo de mercado provee financiamiento
por medio de la emisión de acciones y permiten luego el intercambio de estas. La aplicación más directa de este tipo de mercados, son las bolsas de valores, cuyo
origen se remonta a finales del siglo XV en las ferias medievales de Europa.

Las bolsas de valores se pueden definir como mercados organizados y especializados, en los que se pueden realizar transacciones de títulos de valores por
medio de intermediarios autorizados. Estas bolsas ofrecen al público y a sus miembros facilidades, mecanismos e instrumentos técnicos que facilitan la negociación
de títulos de valores susceptibles de ofertas públicas, a precios determinados mediante subasta.

La principal función de las Bolsas de Valores implican proporcionar a los participantes información verar, objetiva, completa y permanente de los valores
y las empresas inscritas en la Bolsa, sus emisiones y las operaciones que en ella se realicen, supervisión de actividades.

Las componentes de este sistema son los activos, instituciones financieras cuya misión es contactar demandantes y oferentes en los mercados donde se negocian
los diferentes instrumentos o activos financieros.

Dentro de los estudios de la economía, se habla de que este tipo de mercado es de competencia perfecta, es decir posee características
como: \cite{mankiw2011principles}:
\begin{itemize}
        \item Existe un elevado número de compradores y vendedores. La decisión individual de cada uno de ellos ejercerá escasa influencia sobre el mercado global.
        \item Homogeneidad de los productos, es decir, no existen diferencias entre productos que venden los oferentes.
        \item Transparencia del mercado. TOdos los participantes tienen pleno conocimiento de las condiciones generales en que opera el mercado.
        \item Libertad de entrada y salida de empresas. Todos los participantes, cuando lo deseen, podrán entrar o salir del mercado a costos nulos o casi nulos.
\end{itemize}

\section{Series de tiempo financiera}

Las series financieras tienen características singulares en relación a otras series macroeconómicas,
como son: la mayor frecuencia en la observación de los datos, semanal, diaria o incluso minuto
a minuto, y sobre todo la presencia de heterocedásticidad, que hace inadecuados los modelos
desarrollados para series estacionarias. Estas características diferenciadoras han propiciado nu-
merosos trabajos en las áreas de econometría y economía financiera desde los años 70. Así, en
el área financiera las evidencias sobre estructuras de volatilidades implícitas! ha conducido a
la formulación de modelos de valoración continuos con volatilidad no constante2 , y a su dis-
cretización y estimación como series temporales 3 . Los modelos de volatilidad estocástica que
estudiaremos en este artículo pueden obtenerse como resultado de estas discretizaciones. Por
otra parte, en el área econométrica, el modelo ARCH propuesto por Engle(1982) para modeli-
zación de series heterocedásticas, por sus atractivas propiedades estadísticas, que le permiten
captar las características más habituales de estas series y por la sencillez de su estimación, marca
el punto de partida de estos modelos y sus generalizaciones que tienen también su reflejo en el
,área financieré.


\section{Cómo se generan y sus características}
\section{High frequency}
