\parindent=1.5em

Como se sabe, la tecnología se está haciendo presente en todas y cada una de
las áreas de investigación, como física, química, biología, computación, y lo
que es de interés para este documento es el área de los mercados financieros
\cite{watsham1997quantitative}.

El Algorithmic Trading y el High Frequency Trading son moneda común en los
mercados de capitales desarrollados de Estados Unidos y Europa, llegando a
representar más del 60\% del volumen operado en distintas clases de activos. La
tendencia indica que esta proporción de trading automatizado seguirá creciendo,
dejando atrás al trading manual, donde los robots operaran entre si, removiendo
al elemento humano de las operaciones directas, relegándolo al elemento
comercial y al mantenimiento de estos sistemas y algoritmos, cambiando así la
posición central del trader hacia matemáticos y programadores que desarrollen
algoritmos eficaces y eficientes para la compra y venta de activos.

Esta memoria está enfocada a abordar un problema relacionado con las series
financieras de alta frecuencia y una forma particular de poder realizar
pronósticos tomando en cuenta las distintas características de naturaleza
propia de este tipo de datos. Se pretende abordar esta problemática con
metodologías computacionales, aplicando análisis matemáticos útiles para esta
área. El problema es de carácter financiero, por lo que es necesario
contextualizar el tema mediante conceptos, criterios y términos generales
asociados al área.

Se abordará el problema de predicción del valor de una cartera de activos
financieros con el Modelos de Vectores de Corrección del Error (VEC por sus
siglas en inglés). El principio detrás de este modelos es que existe una
relación de equilibrio a largo plazo entre variables económicas y que, sin
embargo, en el corto plazo puede haber desequilibrios. Con los modelos de
corrección del error, una proporción del desequilibrio de un período (el error,
interpretado como un alejamiento de la senda de equilibrio a largo plazo) es
corregido gradualmente a través de ajustes parciales en el corto plazo.  Los
parámetros de este modelo son obtenidos usando el método mínimos cuadrados.
Siendo este factor la motivación principal, ya que este involucra una serie de
cálculos, estimaciones y en sí, es computacionalmente costoso ya que
dependiendo de ciertos factores, como la cantidad de activos, se puede estar
trabajando con variables de gran tamaño (matrices). Puesto que en este tipo de
mercado la eficiencia es un factor clave, se pretende implementar el algoritmo
usando técnicas de High Performance Computing y Graphical Processing Unit.

Los \emph{\textbf{objetivos principales}} de esta memoria son:
\begin{itemize}
	\item Implementar un modelo VEC y analizar su comportamiento con
series financieras de alta frecuencia.
	\item Optimizar dichos algoritmos usando computación CUDA para aumentar su rendimiento.
\end{itemize} 

Los \emph{\textbf{objetivos secundarios}}:
\begin{itemize}
	\item Adaptar el modelo VEC para este tipo serie, encontrando
los parámetros apropiados para su funcionamiento.
	\item Evaluar la factibilidad de optimizar el modelo VEC mediante la programación en CUDA.
\end{itemize}


\textbf{Organización de la Memoria}

Esta memoria se organizará con el siguiente esquema:
\begin{itemize}
    \item Capitulo 2: High Frequency Trading: se explicarán las principales
    componentes de este tipo de mercado
	\item Capitulo 3: High Performance Computing: reseña general a HPC y el 
	lenguaje de programación en CUDA.
    \item Capítulo 4: Estado del arte: se estudiarán y repasarán las técnicas
    relativas a .
    \item Capítulo 5: Descripción formal del problema: se formalizará el
    problema a resolver, indicando las características consideradas.
    \item Capítulo 6: Solución propuesta: se definirá una propuesta de
    solución, con la respectiva metodología de implementación. 
    \item Capítulo 7: Estudio experimental: se implementará y testeará la
    solución propuesta, comparando sus resultados.
    \item Capítulo 8: Conclusiones: se realizarán conclusiones generales del
    trabajo realizado y se detallarán posibles trabajos futuros.
\end{itemize}
