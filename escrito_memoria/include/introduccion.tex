\parindent=1.5em

A través de los años el hombre ha presentado un cambio sustancial en su nivel
de vida; los conocimientos que ha logrado acumular y aplicar han sido para su
beneficio, y han cambiado radicalmente su modo de vivir. Existe una notable
diferencia entre el hombre de hace unas cuantas décadas y el hombre moderno.
Tal diferencia se ha dado por el desarrollo de la ciencia, que está
estrechamente relacionada con la innovación tecnológica. Por esta razón se
amplía el contenido de cómo ha evolucionado la ciencia y la tecnología en el
mundo, su origen remoto, los países que más han aportado en esta área y su
respectiva utilización, ya sea para el desarrollo o la destrucción. Como se
sabe, la tecnología se está haciendo presente en todas y cada una de las áreas
de investigación, como física, química, biología, computación, y lo que es de
interés para este documento es el área de los mercados financieros
\cite{watsham1997quantitative}.

El Algorithmic Trading y el High Frequency Trading son moneda común en los mercados de capitales 
desarrollados de Estados Unidos y Europa, llegando a representar más del 60\% del 
volumen operado en distintas clases de activos. La tendencia indica que esta proporción 
de trading automatizado seguirá creciendo, dejando atrás al trading manual, donde 
los robots operaran entre si, removiendo al elemento 
humano de las operaciones directas, relegándolo al elemento comercial y al mantenimiento de 
estos sistemas y algoritmos, cambiando así la posición central del trader hacia 
el Matemáticos y Programadores que desarrollen algoritmos eficaces y eficientes 
para la compra y venta de activos.

Esta memoria está enfocada a abordar un problema relacionado con las series
financieras de alta frecuencia y una forma particular de poder realizar
pronósticos tomando en cuenta las distintas características de naturaleza
propia de este tipo de datos. Se pretende abordar esta problemática con
metodologías computacionales, aplicando algunos análisis matemáticos útiles
para esta área. El problema es de carácter financiero, por lo que es necesario
contextualizar el tema mediante conceptos, criterios y términos generales
asociados al área.

Dadas las características particulares que presentan las series financieras de
alta frecuencia, su análisis se puede transformar en una tarea árdua. La
literatura presenta distintos modelos y métodos para resolver este problema,
sin embargo no existe un método único que pueda solucionar este tipo problema.
Los métodos actuales inclusive, no consideran todas las características propias
de este tipo de fenómeno. %La no existencia de un método que solucione este tipo
%de problema, ha generado una tendencia de fusionar métodos, con el objetivo de
%abodar de mejor forma el problema.

%La motivación principal de esta memoria, es que el modelo no ha sido aplicado
%para serie financieras de alta frecuencia, y en los casos que se ha
%implementado, si bien ha tenido buenos resultados, no ha logrado tener buen
%rendimiento a nivel de eficiencia, ya que el proceso involucra una alta
%cantidad de cálculos. En base a esto se identificaron los objetivos para esta
%memoria.

%La mayor parte de los métodos, analizan este tipo de fenómenos en el dominio
%del tiempo, lo que no ha presentado buenos resultados.  Uno de los enfoques, es
%realizar análisis de Multiresolución Wavelet (MRA) \cite{benaouda2006wavelet},
%centrando el estudio en el dominio de la frecuencia.  Este enfoque se ha
%utilizado en distintas áreas de la ciencia, y en particular en series
%financieras, teniendo resultados alentadores. Más en particular, Zhang ha
%propuesto un modelo neural-wavelet \cite{zhang2001adaptive}, mezclando redes
%neuronales con MRA, los cuales llegan a buenas aproximaciones pero no logran
%competir en eficiencia temporal. La idea es poder rescatar información de la
%serie por bandas, para luego realizar pronósticos (por bandas) mediante redes
%neurnales, y se espera, que la predicción de multiple escala tenga mejor
%performance de pronóstico que solamente una red. 

Los \emph{\textbf{objetivos principales}} de esta memoria son:
\begin{itemize}
	\item Implementar un modelo X y analizar su comportamiento con
series financieras de alta frecuencia.
	\item Optimizar dichos algoritmos usando computación CUDA para aumentar su rendimiento.
\end{itemize} 

Los \emph{\textbf{objetivos secundarios}}:
\begin{itemize}
	\item Adaptar el modelo X para este tipo serie, encontrando
los parámetros apropiados para su funcionamiento.
	\item Evaluar la factibilidad de optimizar el modelo X mediante la programación en CUDA.
\end{itemize}

\section{Organización de la Memoria}

Esta memoria se organizará con el siguiente esquema:
\begin{itemize}
	\item Capitulo 2: High Frequency Trading: se explicarán las principales 
	componentes de este tipo de mercado
	\item Capitulo 3: High Performance Computing: reseña general a HPC y el 
	lenguaje de programación en CUDA.
	\item Capítulo 4: Estado del arte: se estudiarán y repasarán las técnicas
relativas a .
	\item Capítulo 5: Descripción formal del problema: se formalizará el
problema a resolver, indicando las características consideradas.
	\item Capítulo 6: Solución propuesta: se definirá una propuesta de
solución, con la respectiva metodología de implementación. 
	\item Capítulo 7: Estudio experimental: se implementará y testeará la
solución propuesta, comparando sus resultados.
	\item Capítulo 8: Conclusiones: se realizarán conclusiones generales del
trabajo realizado y se detallarán posibles trabajos futuros.
\end{itemize}
