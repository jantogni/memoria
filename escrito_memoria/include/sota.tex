\section{Modelos de pronósticos clásicos}

En este capítulo se estudiarán los modelos clásicos de pronóstico, y se subdividirán en métodos lineales y no lineales.

\subsection{Lineales}

\subsubsection{Estacionalidad}

\subsubsection{Auto Regressive (AR)}

Un modelo autorregresivo (AR) es un tipo de proceso aleatorio que se utiliza a menudo para modelar y predecir diversos tipos de fenómenos naturales. 
Este modelo es uno de un grupo de fórmulas de predicción lineal que tratan de predecir una salida de un sistema basado en las salidas anteriores.

\subsubsection{Moving Average (MA)}

Mientras los modelos AR estiman qué proporción de los datos del período pasado es probable que persistan en períodos futuros, los modelos MA se centran
en cómo los datos futuros reaccionan a la innovación en los datos anteriores.

\subsubsection{Auto Regressive Moving Average (ARMA)}
\subsubsection{Auto Regressive Integrated Moving Average (ARIMA)}

\subsection{Volatilidad}
\subsubsection{Auto Regressive Conditional Heteroskedasticity (ARCH)}
\subsubsection{Generalized Auto Regressive Conditional Heteroskedasticity (GARCH)}

\subsection{No Lineales}
\subsection{Taylor series expansion (bilinear models)}
\subsection{Threshikd autoregressive model}
\subsection{Markov switching model}
\subsection{Nonparametric estimation}
\subsection{Neural networks}

\section{Modelos de pronóstico multiresolución}
