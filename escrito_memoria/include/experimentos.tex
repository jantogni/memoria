\section{Selección de data (activos, frecuencia)}

Los datos para pruebas y experimentos se descargaron desde la plataforma
Dukascopy. Incialmente las pruebas se realizaron con datos a frecuencia de 1
minuto, pero luego se implementó en clase Reader los métodos correspondientes
para manejar la data tick de forma automática: juntar datos de distintas
monedas (Bid o Ask), hacer un resampling a diferentes frecuencias, normalizar
las monedas dejando siempre la misma base, etc. Las divisas elegidas para
realizar los experimentos fueron: \emph{EURUSD}, \emph{GBPUSD}, \emph{USDCHF},
\emph{USDJPY} y se trabajó con el Ask Price \ref{fig:stocks_ask}. En el caso de
\emph{USDCHF}, \emph{USDJPY} se trabajó con su recíproco, para que todos los
cálculos quedaran en la misma base \emph{USD}. Además la frecuencia, muestreada
en minutos, genera una ventana de 1440 datos por día.

\begin{figure}[h!t]
    \begin{center}
        \includegraphics[width=\textwidth]{images/stocks_ask}
        \caption{Precio Ask de las divisas}
        \label{fig:stocks_ask}
    \end{center}
\end{figure}

Los ticks están compuestos por: \emph{Date}, \emph{Time}, \emph{Ask},
\emph{Bid}, \emph{AskVolume}, \emph{BidVolume}(ver tabla~\ref{tab:ticks}). Por
ser datos ticks, el campo Time es un hora \emph{double}, es decir, tiene
asociada una hora :minutos: segundos, donde segundos es un número no entero.
Cabe destacar que no existe una clara relación entre la aparición de ticks, ya
que en algunos casos aparecen hasta 4 ticks en 1 segundo, mientras que en otros
horarios no aparecen ticks en 90 segundos (dato calculado con la función
check\_min\_frequency de la clase Reader). En algunos casos, cómo en los días
más emblemáticos del año, que la cantidad de ticks por día incrementa
considerablemente, por ejemplo: un día reglar tiene aproximadamente 35 mil
ticks, mientras que días como 11 de Septiembre que hay 87 mil ticks. 

\begin{table}[h!]
\caption{Data Tick}
\label{tab:ticks}
\begin{center}
\begin{tabular}{|c|c|c|c|c|c|}
\hline
Date & Time & Ask & Bid& AskVolume & BidVolume \\
\hline
11-08-2014 & 00:00:00.000 & 1.34046 & 1.34042 & 1.25 & 1.69 \\
11-08-2014 & 00:00:02.159 & 1.34047 & 1.34043 & 4.69 & 1 \\
11-08-2014 & 00:00:02.667 & 1.34046 & 1.34042 & 1.32 & 2.44 \\
11-08-2014 & 00:00:03.175 & 1.34046 & 1.34043 & 1.32 & 1 \\
11-08-2014 & 00:00:07.058 & 1.34046 & 1.34043 & 3.75 & 1.69 \\
11-08-2014 & 00:00:07.362 & 1.34043 & 1.34041 & 2.25 & 1 \\
\hline
\end{tabular}
\end{center}
\end{table}
\newpage
En la figura \ref{fig:eurusd_ticks} se puede apreciar la data tick del EURUSD,
en una ventana de 30 minutos, donde la curva superior es la correspondiente al
Ask, y la inferior al Bid. En la figura \ref{fig:eurusd_freq} se muestra la
misma moneda pero con resamples a 1, 10, 30 y 60 segundos. El resample se
calcula con el promedio de los ticks en la ventana de 6 segundos, y para el
caso que no exista movimiento en dicho periodo, se rellena con el valor
anteior.

\begin{figure}[h!t]
    \begin{center}
        \includegraphics[width=0.7\textwidth]{images/eurusd}
        \caption{Ticks de EURUSD, Ventana de 30 minutos}
        \label{fig:eurusd_ticks}
    \end{center}
\end{figure}

\begin{figure}[h!t]
    \begin{center}
        \includegraphics[width=\textwidth]{images/resample_freq}
        \caption{Ticks de EURUSD, Ventana de 30 minutos}
        \label{fig:eurusd_freq}
    \end{center}
\end{figure}

%\begin{figure}[h!t]
%    \begin{center}
%        \includegraphics[width=0.6\textwidth]{images/eurusd_30s}
%        \caption{Ticks de EURUSD con resample de 30 segundos}
%        \label{fig:eurusd_30s}
%    \end{center}
%\end{figure}

\newpage
\section{Selección de Parámetros}
Cómo se mencionó anteriormente, los parámetros del algoritmo se calcularon
mediante el criterio de información de Akaike (AIC). Para esto, se tomaron
datos intradiarios de agosto 2014, semana entre 11-15. Además, se
combinaron:
\begin{itemize}
 \item L: [100, 400, 700, 1000].
 \item P: [1, 2, 3, 4, 5].
 \item Frecuencias: 1 minuto.
\end{itemize}

Los resultados~\ref{tab:IAC} muestran para cada largo de ventana,
el número de lag óptimos (en negrita) para el modelo según el criterio de AIC.

\begin{table}[h]
\caption{Resultados de AIC}
\label{tab:IAC}
\begin{center}
\begin{tabular}{|l|c|c|c|c|c|}
\hline
\backslashbox{\textbf{L}}{\textbf{P}} & \textbf{1} & \textbf{2} & \textbf{3} & \textbf{4} & \textbf{5} \\
\hline
100 & -58.64574 & \textbf{-58.83743} & -58.70088 & -58.61984 & -58.68117 \\
400 & -60.42127 & -60.65667 & -60.67377 & -60.64634 &  \textbf{-60.68087} \\
700 & -59.05503 & -59.17413 & \textbf{-59.20871} & -59.18571 & -59.17367 \\
100 & -58.98121 & -59.09158 & \textbf{-59.12083} & -59.10099 & -59.0927 \\
\hline
\end{tabular}
\end{center}
\end{table}

\section{Pruebas de Cointegración}
Para los efectos de pruebas como se trabajó con 4 monedas, la cantidad de 
vectores de cointegración debería ser 1, 2 o 3. 

\section{Test de raiz unitaria}
Antes de correr los test, se checkearon que las series de tiempo fueran
I(1) usando el test de Augmented Dickey Fuller (ADF).

\begin{table}[h!]
\caption{Test de raiz unitaria}
\label{tab:adf}
\begin{center}
\begin{tabular}{|l|c|c|c|c|c|}
\hline
& \textbf{Estadístico} & \textbf{Valor Crítico} & \textbf{Resultado}\\
\hline
EURUSD          & -0.64 & -1.94 & True       \\
$\Delta$ EURUSD & -70.45   & -1.94 & False       \\
GBPUSD          & -0.63   & -1.94 & True          \\
$\Delta$ GBPUSD & -54.53   & -1.94 & False       \\
CHFUSD          & -0.88   & -1.94 & True         \\
$\Delta$ CHFUSD & -98.98   & -1.94 & False       \\
JPYUSD          & -0.65 & -1.94 & True        \\
$\Delta$ JPYUSD & -85.78 & -1.94 & False     \\ 
\hline
\end{tabular}
\end{center}
\end{table}

Tabla~\ref{tab:adf} muestra que todas las monedas no pueden rechazadas por el test
de raiz unitaria pero si son rechazadas para sus primeras diferencias. Esto significa
que todas las series de tiempo son I(1) por lo que es posible usar los modelos VECM
y OVECM.

\section{Performance}
Dentro de todos los experimentos realizados, según la selección de parámetros
y considerando 400 iteraciones los resultados de tiempos de ejecución se pueden
ver en la tabla~\ref{tab:extimes}

\begin{table}[h!]
\caption{Execution times}
\label{tab:extimes}
\begin{center}
\begin{tabular}{|l|c|c|c|c|c|}
\hline
& L & model params & e  & Time[s] \\
\hline
OVECM & 100 &p=2  & 0      & 2.492\\
OVECM & 100 &p=2  & 0.0026  & 1.606\\
SLVECM & 100 &p=2& -- & 2.100\\
\hline
OVECM & 400 & p=5  & 0      & 3.513\\
OVECM & 400 &p=5  & 0.0041  & 2.569\\
SLVECM & 400 & p=5 & -- & 3.222\\
\hline
OVECM & 700 &p=3  & 0      & 3.296\\
OVECM & 700 &p=3  & 0.0032  & 2.856\\
SLVECM & 700 &p=3 & -- & 3.581\\
\hline
OVECM & 1000 & p=3 & 0      & 4.387\\
OVECM & 1000 & p=3  & 0.0022  & 2.408\\
SLVECM & 1000 & p=3  & -- & 3.609\\
\hline
\end{tabular}
\end{center}
\end{table}

OVECM recibe como parámetro el umbral de error para cambiar los vectores de
cointegración. Es decir, OVECM con e = 0 es lo mismo que ejecutar el SLVECM,
ambos tienen exactamente los mismos vectores de cointegración y estadísticos.

Para ver en detalle qué parte del código demora más se presenta el siguiente
profile de ejecución:
\begin{verbatim}
   Time %Time  Line Contents
============================
               def OVECM(y, L, P, it, avg_error, r, n):
      4   0.0      it2, l = y.shape
               
      7   0.0      dy_true = np.zeros([it, l], dtype=np.float32)
      4   0.0      dy_pred = np.zeros([it, l], dtype=np.float32)
      4   0.0      mape = np.zeros([it, l], dtype=np.float32)
      4   0.0      mae = np.zeros([it, l], dtype=np.float32)
      4   0.0      rmse = np.zeros([it, l], dtype=np.float32)
               
      8   0.0      m = cl.Matrix()
               
      3   0.0      start_time = time.time()
               
    911   0.0      for i in range(it):
 110712   1.6          y_i = y[i:i + L]
               
   1031   0.0          if i == 0:
                           # Initialization
   6839   0.1              beta = m.get_johansen(y_i.as_matrix(), 
                                    P, r)
  29075   0.4              A, B = m.vec_matrix(y_i, P, 
                                    beta.evecr)
                       else:
                           # Update & Out-of-sample forecast
4456245  66.1              A, B = m.vec_matrix_online(A, B, y_i, 
                                    P, beta.evecr)
   6820   0.1              dy_true[i-1, :] = B[-1,:]
  11775   0.2              dy_pred[i-1, :] = np.dot(A[-1,:], x)
               
                       # OLS
 439061   6.5          x, residuals, rank, s = np.linalg.lstsq(A, B)
  79004   1.2          Ax = np.dot(A, x)
               
                       # Internal mape
  28364   0.4          y_true = y_i.as_matrix()[-n:]
  23996   0.4          y_pred = Ax[-n:] + y_i.as_matrix()[-n - 1:-1]
               
                       # Stats info
 512572   7.6          stats = cl.stats(y_true, y_pred)
   8587   0.1          mape[i, :], mae[i,:], rmse[i,:] = stats.mape, 
                            stats.mae, stats.rmse
  36757   0.5          avg_mape = np.average(mape[i, :])
               
                       # Beta update
   2224   0.0          if avg_mape > avg_error:
 865051  12.8              beta = m.get_johansen(y_i.as_matrix(), P)
    307   0.0              r = beta.r
  41151   0.6              A = m.vec_matrix_update(A, y_i, P, 
                                beta.evecr)
  79857   1.2              x, residuals, rank, s = np.linalg.lstsq(A, B)
               
                   # model_it object
   2945   0.0      oecm = cl.model_it(y[L:L + it], dy_pred + 
                        y[L - 1:L + it - 1], dy_true, dy_pred, 
                        mape, mae, rmse)


 Time  Per Hit   % Time  Line Contents
======================================
                         @do_profile(follow=[OVECM, OVECMRR])
                         def main():
    2      2.0      0.0      path = '../data_csv/data_ticks/august_11/'
    2      2.0      0.0      assets = ['EURUSD', 'GBPUSD', 
                                    'USDCHF', 'USDJPY']
                             
    6      1.2      0.0      data_list = [path+ i +'.csv' for i in assets]
                             
    7      7.0      0.0      reader = cl.Reader(data_list)
 100M   100M.0     93.7      ticks_list = reader.load()
                         
39925  39925.0      0.0      ask_norm = reader.resample_list(ticks_list, 
                            '60s', [2, 3])            
    2      2.0      0.0      y = ask_norm
                         
    1      1.0      0.0      L = 1000
    1      1.0      0.0      P = 4
    1      1.0      0.0      it = 400
    1      1.0      0.0      r = 0
    0      0.0      0.0      n = 50
    1      1.0      0.0      avg_error = 0.002
                         
 6758K 6758K.0      6.3      oecm = OVECM(y, L, P, it, avg_error, r, n)

\end{verbatim}

Se puede observar que los cálculos más costosos son la construcción de la
matriz VECM, seguido por el cálculo de los vectores de cointegración, luego de
los estadísticos en cada paso y finalmente la resolución del sistema mediante
mínimos cuadrados. Por otro lado, el cálculo está inserto dentro de la lectura
de archivos, los cuales toman mucho más tiempo que la ejecución del algoritmo.

Cabe destacar que estos números dependen del tamaño de la ventana, número de
lags, umbral de error y cantidad de elementos para calcular el mape in-sample.

En todos los casos, ya sea incluso en frecuencias de 1 segundo, el algoritmo
OVECM, toma tiempos de ejecución menor a la frecuencia de datos, esto le
permite a un algoritmo de estrategia poder utilizar como entrada los resultados
de esta implementación.

\section{Resultados}
Uno de los objetivos del algoritmo OVECM es ser más eficiente en tiempo 
que el SLVECM, pero además se busca que los errores de la extrapolación
no sean tan distintos. La tabla~\ref{tab:mapes} muestra diferentes los diferentes
indicadores de las muestras en la interpolación (in-sample) y la extrapolación 
(out-of-sample). Se puede ver que son similares, por lo que se confirma que los
vectores de cointegración se pueden mantener en cierta medida durante el tiempo.

\begin{table*}[ht!]
\caption{Métricas del modelo}
\label{tab:mapes}
\begin{center}
\begin{adjustbox}{max width=\textwidth}
\begin{tabular}{|l|l|c|c|c|c|c|c|c|c|}
\hline
\multicolumn{4}{|c|}{Model} & \multicolumn{3}{|c|}{In-sample} &
\multicolumn{3}{|c|}{Out-of-sample} \\ 
\hline
\hline
Método & L & P & e &
MAPE & MAE& RMSE&
MAPE & MAE& RMSE \\
\hline
 OVECM  &   100  &  P=2& 0.0026  &  0.00263&  0.00085&  0.00114&  0.00309&  0.00094&  0.00131\\
 OVECM  &   400  &  P=5& 0.0041  &  0.00378&  0.00095&  0.00127&  0.00419&  0.00103&  0.00143\\
 OVECM  &   700  &  P=3& 0.0032  &  0.00323&  0.00099&  0.00130&  0.00322&  0.00097&  0.00132\\
 OVECM  &   1000 &  P=3& 0.0022  &
 \textbf{0.00175}&  \textbf{0.00062}&  \textbf{0.00087} &
 \textbf{0.00172}&  \textbf{0.00061}&  \textbf{0.00090}\\
\hline
 SLVECM  &   100 &  P=2& -  &  0.00262&  0.00085&  0.00113&  0.00310&  0.00095&  0.00132\\
 SLVECM  &   400 &  P=5& -  &  0.00375&  0.00095&  0.00126&  0.00419&  0.00103&  0.00143\\
 SLVECM  &   700 &  P=3& -  &  0.00324&  0.00099&  0.00130&  0.00322&  0.00098&  0.00132\\
 SLVECM  &   1000 &  P=3& -  &
 \textbf{0.00174}&  \textbf{0.00061}&  \textbf{0.00087}&
 \textbf{0.00172}&  \textbf{0.00061}&  \textbf{0.00090}\\
\hline
\end{tabular}
\end{adjustbox}
\end{center}
\end{table*}

En la figura~\ref{fig:accuracy} se puede observar qué tan bien aproxima el
método OVECM para cada divisa. En la figura~\ref{fig:mapes} se puede observar
cuanto es el error de aproximación medido en MAPE para cada iteración.

\begin{figure}[h!t]
    \begin{center}
        \includegraphics[width=\textwidth]{images/y_vs_yhat}
        \caption{Divisas: Valor real y aproximación}
        \label{fig:accuracy}
    \end{center}
\end{figure}


\begin{figure}[h!t]
    \begin{center}
        \includegraphics[width=\textwidth]{images/ovecm_mapes}
        \caption{Divisas: Valor del MAPE en cada iteración}
        \label{fig:mapes}
    \end{center}
\end{figure}
