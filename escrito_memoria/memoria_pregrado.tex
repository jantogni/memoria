%\documentclass[a4paper,10pt]{book}
%\usepackage[utf8x]{inputenc}

\documentclass[12pt,letterpaper]{report}
\usepackage[utf8]{inputenc}
%\usepackage{geometry,fancyhdr}

% \usepackage[spanish]{babel}

\usepackage{subfigure}
\usepackage{algorithm}
\usepackage{algorithmic}
\usepackage{listings}  
\usepackage{amsthm}
\usepackage{textcomp}
\usepackage{multicol}
\usepackage{float}
\usepackage{url}
\usepackage{enumerate}
\usepackage{newlfont}
\usepackage{psfrag}
\usepackage{charter}
\usepackage{setspace}
\usepackage{longtable}
\usepackage{subfigure}
\usepackage{tabularx}
\usepackage[dvips]{epsfig}
\usepackage[centertags]{amsmath}
\usepackage[size=normalsize]{caption}
\usepackage{newlfont}
\usepackage{psfrag}
\usepackage[spanish]{babel}

%fonts matematicos
\usepackage{amsfonts}
\usepackage{amssymb}

\usepackage{framed}
% colocar un texto dentro de un cuadrado que lo contiene
%\begin{framed}
%\end{framed}

%\usepackage{tikz}

\usepackage{listings}
\usepackage{icomma}

%paquete necesario para incluir imaganes .eps en pdf
\usepackage{epstopdf}

\usepackage{UTFSM_Tesis}

\usepackage[active]{srcltx}
%\usepackage[dvips, pdftitle={Jonathan Antognini memoria}, pdfauthor={Jonathan Antognini}, pdfsubject={Memoria}]{hyperref} 
\usepackage[unicode]{hyperref} 

%Modificar el formato de los capitulos, secciones y subsecciones, ademas modifica la visualizacion en la lista de contenidos.
  %\usepackage{titlesec}
  %\usepackage{titletoc}
 

\hfuzz2pt

\hypersetup{%
pdftitle={Memoria jantogni},
pdfauthor={Jonathan Antognini},
pdfkeywords={memoria, jantogni},
bookmarksnumbered,
pdfstartview={FitH},
urlcolor=cyan,
}%

\hyphenpenalty=5000
  \tolerance=1000
\hyphenation{no-me-cor-tes}


 \oddsidemargin  0cm    % Ancho Legal 21,59cm
 \evensidemargin 0.5cm  % Alto  Legal 35,56cm
 \textwidth      16.5cm
 \topmargin       -1.5cm
 %\voffset        2cm  % Margen superior
 \textheight     22cm
 %  \parindent      4cm
 %\parskip        2ex

\newlength{\defbaselineskip}
\setlength{\defbaselineskip}{\baselineskip}

\newcommand{\setlinespacing}[1]%
           {\setlength{\baselineskip}{#1 \defbaselineskip}}
% \newcommand{\doublespacing}{\setlength{\baselineskip}%
%                            {1.3 \defbaselineskip}}
% \newcommand{\singlespacing}{\setlength{\baselineskip}{\defbaselineskip}}

%Formulas matem?ticas utilizadas en el paper
\newcommand{\bbbr}{\mathbb R}

\newcounter{defcount}
\setcounter{defcount}{0}
\newenvironment{definition}[1]
%%\bf\normalsize
{% This is the begin code \bf
\stepcounter{defcount}{\noindent \bf \normalsize Definition \bf \arabic{defcount}: #1}\nopagebreak[3] \\
\begin{it}
}
{% This is the end code
\end{it}

}

\newcounter{defcount1}
\setcounter{defcount1}{0}
\newenvironment{property}[1]
%%\bf\normalsize
{% This is the begin code \bf
\stepcounter{defcount1}{\noindent \bf \normalsize Property \bf \arabic{defcount1}: #1}\nopagebreak[3] \\
\begin{it}
}
{% This is the end code
\end{it}
}





%Redefine plain page style
% \fancypagestyle{plain}{
% \fancyhf{}
% \renewcommand{\headrulewidth}{0pt}
% \fancyfoot[LE,RO]{\thepage}
% }
% 
% % Code for creating empty pages
% % No headers on empty pages before new chapter
% \makeatletter
% \def\cleardoublepage{\clearpage\if@twoside \ifodd\c@page\else
%     \hbox{}
%     \thispagestyle{plain}
%     \newpage|||
%     \if@twocolumn\hbox{}\newpage\fi\fi\fi}
% \makeatother \clearpage{\pagestyle{plain}\cleardoublepage}
% 
% % Define pagestyle
% \pagestyle{	fancy}
% \fancyhf{}
% \renewcommand{\chaptermark}[1]{\markboth{ \emph{#1}}{}}
% \fancyhead[LO]{}
% \fancyhead[RE]{\leftmark}
% \fancyfoot[LE,RO]{\thepage}


%U1 hasta olympiaCentrodecompras
% tomar micro 60 en el paradero del mall, 3 paraderos mas al norte, himmel.


\setlength\parindent{0pt}

\setcounter{secnumdepth}{5}
\setcounter{tocdepth}{5}

%The first line enables numbering of subsubsections. 
%The second line includes subsubsections in the table of contents.



\begin{document}

%\lstset{language=[ISO]C++}

\title{High frequency trading \& Graphics processing unit}
\author{Jonathan Andrés Antognini Cavieres}
\submitdate{Marzo - 2013}
\copyrightyear{2012}
%%%%%%%%%%%%%% Elegir uno de los siguientes grados:
%\ingej
\ingciv
%\phd

% \phd
%%%%%%%%%%%%%%
\profguia{Dr. Luis Salinas Carrasco}      % Cambiado a partir de supervisor
\profcorr{Dr. H\'ector Allende Olivares}         % Cambiado a partir de firstreader

%%%% dedicatorias al termino de la tesis %%%%
\dedicate{
 ASD ASD ASD
}

\nolistoffigures
\nolistoftables

\beforepreface


%  \listoftables	
%  \listoffigures
%  \listofalgorithms

%\input{agradecimientos.tex}

%%\begin{spacing}{.97}
\vspace*{2cm}
\thispagestyle{empty}
%\begin{center}
{\bfseries \Huge Abstract }
\vspace{1.5cm}

\begin{quotation}
The technical analysis of stock markets and foreign exchange market has been
used in order to forecast the behavior of the prices of a certain stock
or currency. Such analysis is based on the detection patterns of
a financial time series, which are known by its non-stationary and random
behavior.

%El análisis técnico de los mercados bursátiles y de divisas se ha utilizado
%para intentar predecir el comportamiento futuro de los precios de un activo o
%una divisa. Dicho análsis se basa en la detección de patrones en una serie
%temporal financiera, las que son conocidas por su comportamiento no
%estacionario y comportamiento aleatorio. 

A interest pattern to address is the integration and cointegration between the
prices of certain stocks or currencies, which establish that when two or more
time series are cointegrated, and considering that both processes are
non-stationary, (as in the case of financial time series), there is a long term
equilibrium relation which link both series such that the relation is
stationary.  The model that take advantage of this scenario is the Vector Error
Correction (VECM)

%Un patrón de interés de estudio es la integración y cointegración entre los
%precios de ciertas divisas. Cuando dos o más series de tiempo están
%cointegradas, a pesar de que ambos procesos no sean estacionarios, como es el
%caso de las series de tiempo financieras, existe una relación de equilibrio a
%largo plazo que vincula ambas series tal que esta relación sea estacionaria. El
%modelo que permite aprovechar este partón es el Vector Error Correction Model
%(VECM).

The cointegration of pairs of currencies are modeled in several works, particularly
with FOREX data, which through software tools such as Matlab, Eviews, etc.
it is possible to computationally forecast the behavior of those time series.

%Existen varios trabajos donde modelan este problema con pares de monedas,
%particularmente con las series de tiempo del Foreign Exchange Market (FOREX).
%Además mediante herramientas computacionales como matlab, eviews, etc. es
%posible predecir computacionalmente el comportamiento de dichas series de
%tiempo.

However, an important aspect of this kind of series are their Online nature,
since every second more information is generated due the Ask or Bid orders,
topic which had been not addressed completely.  VECM is a matricial model,
which from the Online point of view, suffers of values updates, which implies
the solution of a matricial system equation, to be able to forecast.

%Sin embargo un foco importante de este tipo de series es su caracter
%\emph{Online}, ya que a cada segundo se genera nueva información en el sistema
%debido a la compra o venta de divisas. Tema que no ha sido abordado
%completamente en la literatura. VECM es un modelo matricial, que desde el punto
%de vista \emph{Online}, sufre ciertas actualizaciones de valores, y para poder
%predecir es necesario resolver un sistema matricial.

On this work, we present the VEC, Sliding Windows VEC and Online VEC Models.
The objective of the Online VECM is to optimize the computational calculations,
reducing the number of times that the cointegration vectors are calculated.
Additionally, a class diagram and their implementation associated to Use Cases
was developed, considering: data reading, data re-sample in certain frequency,
graphics, metrics, etc.

%En este trabajo se presenta el modelo VEC, el modelo VEC para ventanas
%deslizantes y la implementación \emph{Online} que busca optimizar cálculos
%computacionales reduciendo la cantidad de veces que se calculan los vectores de
%cointegración. Para ello se creo un diagrama de clases asociado a los casos de
%uso del modelo, considerando entre ello: lectura de datos, resample a cierta
%frecuencia, gráficos de resultados, etc.

We present the result of a set of tests with different currencies, ensuring all
the conditions to apply this model. The results shows that the Online VEC Model
version, reduce almost 50\% the execution time, maintaining the algorithm
accuracy. Additionally, we developed a couple of CUDA routines, in order to
optimize the calculation.

%Se realizaron pruebas con diferentes monedas de FOREX asegurando las
%condiciones para poder aplicar el modelo. Los resultados muestran que la
%versión \emph{Online} del modelo VEC, disminuye cerca del 50\% el tiempo de
%ejecución sin comprometer la precisión del algoritmo. Además se implementaron
%rutinas en CUDA para optimizar los tiempos de ejecución.

{\bf Keywords:} VECM, VAR, FOREX, GPU.
\end{quotation}

\newpage
% Español

\vspace*{2cm}
\thispagestyle{empty}
%\begin{center}
{\bfseries \Huge Resumen }
\vspace{1.5cm}

\begin{quotation}
El análisis técnico de los mercados bursátiles y de divisas ha sido utilizado
para intentar predecir el comportamiento de los precios de un activo o una
divisa. Dicho análsis se basa en la detección de patrones en una serie temporal
financiera, las que son conocidas por su comportamiento no estacionario y
aleatorio. 


Un patrón de interés de estudio es la integración y cointegración entre los
precios de ciertas divisas, el cual establece que cuando dos o más series de
tiempo están cointegradas, y considerando que ambos procesos no sean
estacionarios (como es el caso de las series de tiempo financieras), existe una
relación de equilibrio a largo plazo, que vincula ambas series para que esta
relación sea estacionaria. El modelo que permite aprovechar este partón es el
Vector Error Correction Model (VECM).

Existen varios trabajos donde modelan este problema con pares de monedas,
particularmente con las series de tiempo del Foreign Exchange Market (FOREX),
las cuales mediante herramientas computacionales como Matlab, Eviews, entre
otros, es posible predecir computacionalmente el comportamiento de dichas
series de tiempo.

Sin embargo un foco importante de este tipo de series es su carácter
\emph{Online}, ya que a cada segundo se genera nueva información en el sistema
debido a la compra o venta de divisas, el cual no ha sido abordado
completamente en la literatura.  VECM es un modelo matricial, que desde el
punto de vista \emph{Online}, sufre ciertas actualizaciones de valores, lo que
implica que para poder realizar una predicción es necesario resolver un sistema
matricial.

En este trabajo se presenta el modelo VEC, el modelo VEC para ventanas
deslizantes y la implementación \emph{Online} que busca optimizar cálculos
computacionales, reduciendo la cantidad de cálculo de los vectores de
cointegración.  Complementariamente, se creó un diagrama de clases asociado a
los casos de uso del modelo, considerando entre ello: lectura de datos,
resampleo a cierta frecuencia, gráficos de resultados, etc.

Se realizaron pruebas con diferentes monedas de FOREX, asegurando las
condiciones necesarias para poder aplicar el modelo. Los resultados muestran
que la versión \emph{Online} del modelo VEC, disminuye cerca del 50\% el tiempo
de ejecución sin comprometer la precisión del algoritmo. Adicionalmente, se
implementaron rutinas utilizando CUDA para optimizar el cálculo.

{\bf Palabras claves:} VECM, VAR, FOREX, GPU.

\end{quotation}


\afterpreface

\onehalfspacing

%\input{capitulo_1.tex}
%\input{capitulo_2.tex}
%\input{capitulo_3.tex}

\chapter{Introducción}
	\parindent=1.5em

Como se sabe, la tecnología se está haciendo presente en todas y cada una de
las áreas de investigación, como física, química, biología, computación, y lo
que es de interés para este documento es el área de los mercados financieros
~\citep{watsham1997quantitative}.

El Algorithmic Trading y el High Frequency Trading son moneda común en los
mercados de capitales desarrollados de Estados Unidos y Europa, llegando a
representar más del 60\% del volumen operado en distintas clases de activos. La
tendencia indica que esta proporción de trading automatizado seguirá creciendo,
dejando atrás al trading manual, donde los robots operaran entre si, removiendo
al elemento humano de las operaciones directas, relegándolo al elemento
comercial y al mantenimiento de estos sistemas y algoritmos, cambiando así la
posición central del trader hacia matemáticos y programadores que desarrollen
algoritmos eficaces y eficientes para la compra y venta de activos.

Esta memoria está enfocada a abordar un problema relacionado con las series
financieras de alta frecuencia y una forma particular de poder realizar
pronósticos tomando en cuenta las distintas características de naturaleza
propia de este tipo de datos. Se pretende abordar esta problemática con
metodologías computacionales, aplicando análisis matemáticos útiles para esta
área. El problema es de carácter financiero, por lo que es necesario
contextualizar el tema mediante conceptos, criterios y términos generales
asociados al área.

Se abordará el problema de predicción del valor de una cartera de activos
financieros con el Modelos de Vectores de Corrección del Error (VEC por sus
siglas en inglés). El principio detrás de este modelos es que existe una
relación de equilibrio a largo plazo entre variables económicas y que, sin
embargo, en el corto plazo puede haber desequilibrios. Con los modelos de
corrección del error, una proporción del desequilibrio de un período (el error,
interpretado como un alejamiento de la senda de equilibrio a largo plazo) es
corregido gradualmente a través de ajustes parciales en el corto plazo.  Los
parámetros de este modelo son obtenidos usando el método mínimos cuadrados.
Siendo este factor la motivación principal, ya que este involucra una serie de
cálculos, estimaciones y en sí, es computacionalmente costoso ya que
dependiendo de ciertos factores, como la cantidad de activos, se puede estar
trabajando con variables de gran tamaño (matrices). Puesto que en este tipo de
mercado la eficiencia es un factor clave, se pretende implementar el algoritmo
usando técnicas de High Performance Computing y Graphical Processing Unit.

Los \emph{\textbf{objetivos principales}} de esta memoria son:
\begin{itemize}
	\item Implementar un modelo VEC y analizar su comportamiento con
series financieras de alta frecuencia.
	\item Optimizar dichos algoritmos usando computación CUDA para aumentar su rendimiento.
\end{itemize} 

Los \emph{\textbf{objetivos secundarios}}:
\begin{itemize}
	\item Adaptar el modelo VEC para este tipo serie, encontrando
los parámetros apropiados para su funcionamiento.
	\item Evaluar la factibilidad de optimizar el modelo VEC mediante la
programación en CUDA.
\end{itemize}


\textbf{Organización de la Memoria}

Esta memoria se organizará con el siguiente esquema:
\begin{itemize}
 \item Capitulo 2: High Frequency Trading: En la primera sección se
explicarán las principales componentes de este tipo de mercado. En la
Segunda sección se definirá el problema a abordar.
 \item Capítulo 3: Antecedentes: Se presentan los fundamentos matemáticos y
modelos que se usarán durante el desarrollo de esta memoria.
 \item Capitulo 4: Graphics Processing Units: reseña general a HPC y el 
lenguaje de programación en CUDA.
 \item Capítulo 4: Metodología: se propone un algoritmo VEC y la metodología de
desarrollo. Cómo se seleccionarán los parámetros del modelo y un diagrama de
clase con las implementaciones realizadas.
 \item Capítulo 5: Experimentos: se presentará como se seleccionó la data y
parámetros del algoritmo. Se probará el algoritmo propuesto, y se compararán
los resultados entre las implementaciones realizadas.
 \item Capítulo 6: Conclusiones: se realizarán conclusiones generales del
trabajo realizado y se detallarán posibles trabajos futuros.
\end{itemize}


\chapter{Estado del arte}
	En este capítulo se revisarán las aplicaciones y estudios hechos en el área. Para ordenar esta sección, se
dividirá respecto a los métodos cuantitativos a usar (neural-wavelet), sus aplicaciones en esta área, y finalmente
los desarrollos realizados mediante técnicas de HPC.

\section{Artificial Neuronal Network}

Las redes neuronales artificiales (ANN), han sido usadas en distintos campos de la ciencia. En particular en el área de computación, son de interés
como herramienta para procesos de minería de datos \cite{bigus1996data}, ya que se ha convertido en una metodología multipropósito, robusta computacionalmente, 
con apoyo teórico sólido. Como es conocido los problemas de minería de datos no son de naturaliza computacional generalmente, sino que son técnicas o metodologías, para
satisfacer algún tipo de apoyo a procesos que involucren manejo con grandes volúmenes de datos. Los modelos de redes neuronales buscan encontrar relaciones entre los datos existentes, y
la manera en que lo hacen es de forma inductiva, es decir, mediante algoritmos de aprendizaje.

Una neurona aritificial es un procesador elemental (PE), que recibe una serie de entradas con pesos diferentes, las procesa y proporciona una salida única. A cada neurona llegan muchas señales
de otras, proceso conocido en la biología como Sinapsis, y producen una única salida (Axon). Una sinapsis que comuna dos neuronas puede ser de naturaleza excitadora o inhibidora. En el primer caso, 
la neurona emisora tenderá a activar a la neurona receptora, y en el segundo caso, la neurona emisora tenderá a inhibir la actividad de la neurona receptora.

Cada sinapsis se caracteriza además por la eficacia con la que se establece la conexión. EN definitiva, aunque la neurona toma dediciones en función de la suma de información que recibe, la contribución
de cada una de estas informaciones que recibe, la contribución de cada una de estas finromaciones es ponderada por la eficacia de la sinapsis correspondiente.

La primera neurona artitificial fue concebida por W. McChulloch, y W. Pitts. Se trata de un modelo binario cuyo estado es 1 (activo) o 0 (inactivo). Periódicamente actualiza su estado
calculando la suma de sus entradas con el valor de cada entrada modulado por la eficacia sináptica correspondiente, y toma una decisión comparando esta suma con un cierto nivel fijado. Si
la suma es superior al umbral, la neurona se activa, y en caso contrario inactiva. Por tanto, todas las neuronas toman sus decisiones simultáneamente tendiendo en cuenta la evolución del estado
global de la red. La función de activación corresponde a una función no lineal. El modelo queda como:


$$ y = \gamma \left( \sum_{i = 1}^{m}w_ix_i + w_0 \right) $$

Donde:
\begin{itemize}
	\item $w_i$: pesos de la entrada i o eficacia de la sinapsis.
	\item $x_i$: entrada i.
	\item $w_o$: sesgo.
	\item $\gamma$: función no lineal.
\end{itemize}

La propuesta de McCulloch fue una salida binaria $sign(z)$, es decir:

$$ \gamma(z) = sign(z) = \left\{
	       \begin{array}{ll}
		 1      & \mathrm{si\ } z \ge 0 \\
		 -1	& \mathrm{si\ } z < 0  \\
	       \end{array}
	     \right. $$

Esta no es la única función no lineal que se puede especificar.

\subsection{Tipos de aprendizaje}

Una de las principales capacidades de una ANN es su capacidad de aprender a partir de un conjunto de patrones de entrenamiento, es decir, que es capaz
de encontrar un modelo de ajustes de dato. Es por ello que se conocen varios tipos de aprendizajes:

\subsubsection{Aprendizaje supervisado}

El aprendizaje supervisado es un caso de entrenamiento con entrenador, y se utiliza información global. En su implementación se presentan dos vectores (uno de entrada y otro de salida deseada).
La salida computada por la red se compara con la salida deseada, y los pesos de la red se modificacian en el sentido de reducir el error cometido. Se repite iterativamente, hasta que la diferencia
entre la salida computada y la deseada sea aceptablemente pequeña, comparada con algún parámetro de error. Con \emph{n} parejas de este tipo se forma un conjunto de entrenamiento.

El aprendizaje supervisado se suele dividir a su vez en dos sub categorías:
\begin{itemize}
	\item[-] Aprendizaje estructural: se refiere a la búsqueda de la mejor conexión o afinidad posible entrada/salida para cada paerja de patrones individuales. Este enfoque es uno de los más utilizados
	\item[-] Aprendizaje temporal: hace referencia a la captura de una serie de patrones necesarios para conseguir algún resultado final. En el aprendizaje temporal la respuesta actual de la red depende de las entradas
		y respuestas previas. En el aprendizaje estructural no existe esta dependencia.
\end{itemize}

\subsubsection{Aprendizaje no supervisado}

El aprendizaje no supervisado es un caso de entrenamiento sin entrenador y sólo se usa información local durante todo el proceso de aprendizaje. Es un modelo más cercano al sistema biológico, no se utiliza vector de salida
esperada, y sólo hay vectores de entrada en el conjunto de entrenamiento. El algoritmo modifica los pesos de forma que las salidas sean consistentes, es decir, que a entradas muy parecidas, la red compute la misma salida. 
Las salidas se asocian a las entradas de acuerdo con el proceso de entrenamiento. El proceso extrae características, abstrayendo las propiedades colectivas subyacentes del conjunto de entrenamiento, y agrupa por clases 
de similitudes.

\subsubsection{Aprendizaje Hebbiano}

El aprendizaje Hebbiano propone que los pesos de la red se incrementan si las neuronas origen y destino están activadas, es decir, refueza los caminos usados frecuentemente en la red, lo que explicaría los
hábitos y el aprendizaje por repetición.

El aprendizaje hebbiano está matemáticamente caracterizado por la ecuación:

$$ w_{ij}^{nuevo} = w_{ij}^{anterior} + a_{ki}b_{kj} $$

Donde $i = 1,2,...,n$; $j=1,2,...,p$; $w_{ij}$ es el peso de la conexión entre los dos procesadores elementales (neuronas artificiales).

Las redes neuronales como la memoria asociativa lineal emplean este tipo de aprendizaje. El número de patrones que una red adiestra usando conexiones y pesos ilimitados puede producir, está limitado por la dimensión de 
patrones de entrada.

Si los valores de los PEs están limitados y los pesos ilimitados, se encuentra el caso denominado Hopfield, que restringen el valor de los PEs a un valor binario o bipolar.
producir está

\subsubsection{Aprendizaje competitivo}

El aprendizaje competitivo usa inhibición lateral para activar una sola neurona (se puede ver como el ganador). Algunas redes neuronales que emplean aprendizaje competitivo son los
mapas auto-organizativos (Kohonen,1984) y Adaptive Resonanse Theory (Caprenter y Grossber).

\subsubsection{Aprendizaje Min-Max}

Un clasificador min-max usa un par de vectores para cada clase. La clase \emph{j} está representada por el PE $y_i$ y está definida por los vectores $V_j$ (el vector min) y $W_j$ (el vector max).
El aprendizaje min-max es un sistema neuronal que viene dado por la ecuación:

$$ v_{ij}^{nuevo} = min(a_{ki},v_{ij}^{anterior}) $$

para el vector min y:

$$ w_{ij}^{nuevo} = min(a_{ki},w_{ij}^{anterior}) $$

\subsubsection{Aprendizaje de correción de error}

Este tipo de aprendizaje ajusta los pesos de conexión entre PEs en proporción a la diferencia entre los valores deseados y los computados
para cada PE de la capa de salida. Dependiendo del número de apas de las redes se distinguen dos casos:
\begin{itemize}
	\item[-] Red de dos capas: puede capturar mapeos lineales entre las entradas y salidas. DOs redes neuronales que utilizan este tipo de aprendizaje son el Perceptrón (Rosenblatt) y ADALINE (Widrow y Hoff)
	\item[-] Red multicapa: peden capturar mapeos no lineales entre las entradas y salidas. La versión multinivel de este algoritmo es denominado Regla de Aprendizaje de Retropropagación de errores (Backpropagation).
		Utilizando la regla encadenada, se calculan los cambios de los pesos para un número arbitrario de capas. El número de iteraciones que deben ser realizadas para cada patrón del conjunto de datos es grande, 
		haciendo este algoritmo de aprendizaje muy lento para entrenar. El algoritmo de retropograpagión ha sido estudiado por Werbos (1974) y Parker (1982), y fue introducido por Rumerlhart, Hilton y Williams (1986).
\end{itemize}

\subsubsection{Aprendizaje reforzado}

Esta heurística para redesneuronales fue ideada por Widrow, Gupta y Maitra (1973) y desarrollado por Williams (1983). Este tipo de aprendizaje es similar al anterior, en que los pesos
son fortalecidos en las acciones desarrolladas correctamente y penalizados en aquellas mal realidas. La diferencia entre ambas es que el aprendizaje por correción de error utiliza
información de error más específica reuniendo valores del error por cada PE de la capa de salida, mientras que el aprendizaje reforzado utiliza información de error no específica
para determinar el desarrollo de la red. MIentras que el primero tiene un vector completo de valores que utiliza para la correción de error, sólo un valor es usado para describir la ejecución de la capa
de salida durante el aprendizaje reforzado. Esta forma de aprendizaje es ideal en situaciones donde no está disponible información específica sobre el error, pero sí información global de la ejecución,
tal como predicción y control.

Las redes que implementan este tipo de aprendizaje son: Adaptive Hueristic Critic, Barto, Sutton y Anderon 1983, y Associative Reward-Penalty, Barto 1985.

\subsubsection{Tabla Resumen}

Una tabla de resumen para recordar los factores e importancia de cada tipo de aprendizaje sería:

\begin{tabularx}{\textwidth}{|X|X|X|X|X|X|}
	\hline 
	Aprendizaje	& Tiempo entrenamiento	& Supervisión			& Linealidad	& Estructural / Temporal	& Cap. de almacen. \\
	\hline 
	Hebbiano 	& Rápido		& No supervisado		& Lineal	& Estructural			& Baja	\\
	Competitivo	& Lento			& No supervisado		& Lineal	& Estructural			& Buena	\\
	Min-Max		& Rápido		& No supervisado		& No lineal	& Estructural			& Buena	\\
	Corr. error dos niveles & Lento		& Supervisado			& Lineal	& Ambos				& Buena	\\
	Corr. error multinivel & Muy lento	& Supervisado			& No lineal	& Ambos				& Alta	\\
	Reforzado	& Muy lento		& Supervisado			& No lineal	& Ambos				& Buena \\
	\hline 	
\end{tabularx}


















\subsection{Feedforward Artificial Neuronal Network}

El modelo de red neuronal a utilizar en esta memoria, serán las Feedforward Artificial Neuronal Network (FFANN).

$$ g_{\lambda}(x,w) = \gamma_2 \left( \sum_{j = 1}^{\lambda} w_j^{[2]} \gamma_1 \left( \sum_{i = 1}^{m} w_{ij}^{[1]}x_i + w_{m+1,j}^{[1]} \right) + w_{\lambda+1}^{[2]} \right) $$

En donde:
\begin{itemize}
	\item Las principales componentes se mantienen al igual que en el modelo más simple, es decir, $w$ y $x$ siguen siendo los vectores de los pesos y datos de entrada.
	\item $\gamma_2$: Es una función que puede ser lineal o no.
	\item $\gamma_1$: Es una función no lineal y diferenciable.
\end{itemize}

La estructura física de cómo se compone el modelo es una división por capas:
\begin{itemize}
	\item Capa de entrada:
	\item Capa oculta:
	\item Capa de salida:
\end{itemize}

\section{Wavelet Multiscale}

\section{High Performance Computing}


\chapter{Descripción del problema}
	\section{Descripción histórica}
\section{Descripción y formalización a usar}


\chapter{Solución propuesta}
	\section{Algoritmo y fundamentos teóricos}


\chapter{Estudio experimental}
	\section{Selección de data (sector, frecuencia)}
\section{Parámetros del algoritmo}
\section{Validación}
\section{Speed up: multicore, gpu, híbrido}


\chapter{Conclusiones}
	El presente trabajo de memoria, expone y valida con la ejecución de un conjunto
de pruebas, la aplicación de un  \emph{Online Vector Error Correction Model} a
series financieras de alta frecuencia.

Basándose en los resultados expuestos en el capítulo ~\ref{ch:experimentos}, se
puede observar que OVECM reduce considerablemente los tiempos de ejecución en
comparación al SLVECM sin comprometer la precisión de la solución. OVECM en
comparación con la versión SLVECM reduce el tiempo de ejecución debido
principalmente al ahorro de cómputo de los vectores de cointegración, los
cuales se calculan mediante el método de Johansen. 
La condición para obtener nuevos vectores de cointegración, es una métrica
(MAPE) de la muestra in-sample, la cual estima qué tan bien se ajusta el modelo
a la data real. Por otra parte, OVECM introduce dos funciones de optimización
de cálculo matricial para obtener el modelo de forma iterativa y no construir
el modelo completo en cada paso.  
VECM, tal como otros algoritmos que deben resolver sistemas matriciales, fue
implementado con dos métodos: OLS y el planteado por Coleman, también conocido
como Ridge Regression. El segundo tiene por objetivo evitar problemas que se
puedan generar tanto numéricamente, como también problemas propios de las
matrices (rank-deficient). El tiempo de ejecución de este método depende
directamente de la cantidad de filas, columnas y rank de la matriz, se observó
que en casos reales la implementación en GPU no tiene mejor performance que la
versión en CPU, sin embargo para matrices cuadradas de gran dimensión, la
versión GPU tiene mejor performance.
En cuanto a tiempo de ejecución general, el algoritmo toma mucho menos tiempo
que la frecuencia con la que los datos arriban, esto significa que el resultado
puede ser usado en estrategias de trading.

Por otra parte, en cuanto a implementación, se hizo un trabajo consistente para
futuras implementaciones, esto es, documentación y estructura de clases acorde
a las necesidades del algoritmo. Principalmente el código está hecho en Python,
lo cual genera una barrera de entrada baja al momento de jugar con el código.
Además se crearon Ipython Notebook con ejemplos de cómo usar cada clase,
métodos y algoritmos.

\newpage
\section{Trabajo Futuro}
Finalmente y como trabajo futuro, se propone:
\begin{itemize}
 \item Probar distintos métodos para la selección de parámetros. Para este
trabajo se utilizó Akaike Information Criterion, pero existen otros como
Schwarz Criterion, Hannan-Quinn Criterion, etc.
 \item Buscar cointegración de distintas monedas. En este trabajo se trabajó
únicamente con 4 monedas, por lo que sería interesante buscar otras monedas que
estén cointegradas. Además esto ayudaría a que el modelo tenga más variables
explicativas, con lo cual el modelo podría mejorar.
 \item Para este trabajo solo se utilizó el precio top del order book,
por lo que sería interesante también buscar una forma de incluir el volumen
asociado al precio.
 \item Probar resultados con alguna estrategia. Mediante el protocolo FIX
probar con datos reales, qué bien se ajusta el modelo (perder o ganar).
 \item Implementar el arribo de datos desde un servidor de datos. Actualmente
se baja data histórica de las monedas y se trabaja con ella, no hay conexión
directa con un servidor.
\end{itemize}



\singlespacing
\bibliographystyle{alpha}
%\bibliographystyle{plain}
\nocite{*} %quitar para mostrar s?lo lar referencias citadas.
\bibliography{references}

\end{document}

