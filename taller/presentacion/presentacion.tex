\documentclass{beamer}
\usepackage{etex}
\usepackage{tabularx}
\usepackage[spanish,activeacute]{babel}
\usepackage[utf8]{inputenc}
\usepackage{graphics}
\usepackage{url}
\usepackage{ulem}
\usepackage{beamerthemesplit}
\usepackage{hyperref}
\usepackage{wrapfig}
\usepackage{listings}
\usepackage{color}
\usepackage{adjustbox}

\usepackage{multicol}
%\usepackage{pstricks}
%\usepackage[pdftex]{graphicx}
%\usepackage{amssymb}
%\usepackage{amsmath}
%\usepackage{mathrsfs}
%\usepackage{fancybox}
%\usepackage{fancyvrb}
%\usepackage{fancyhdr}
%\usepackage{tikz}
%\usepackage{index}
%\usepackage{enumerate}
%\usepackage{setspace}
%\usepackage{boxedminipage}
%\usepackage{listings}
%\usepackage{natbib}
%\usepackage{url}
%\usepackage{lastpage}
%\usepackage{color}
%\usepackage{xcolor}
%\usepackage{float}
%\usepackage{algorithm}
%\usepackage{algorithmic}
%\usepackage{slashbox,pict2e}


\usetheme{CambridgeUS}
\usecolortheme{seahorse}

\makeatother
\setbeamertemplate{footline}
{
  \leavevmode%
  \hbox{%
  \begin{beamercolorbox}[wd=.4\paperwidth,ht=2.25ex,dp=1ex,center]{author in head/foot}%
    \usebeamerfont{author in head/foot}Jonathan Antognini, Luis Salinas, Paola Arce%\insertshortauthor
  \end{beamercolorbox}%
  \begin{beamercolorbox}[wd=.6\paperwidth,ht=2.25ex,dp=1ex,center]{title in head/foot}%
    \usebeamerfont{title in head/foot}\insertshorttitle\hspace*{3em}
    \insertframenumber{} / \inserttotalframenumber\hspace*{1ex}
  \end{beamercolorbox}}%
  \vskip0pt%
}
\makeatletter
\setbeamertemplate{navigation symbols}{}

\usepackage{pstricks}
\usepackage{multicol}

\title{High-Frequency Trading \\ \& \\ Graphics Processing Unit}
\subtitle{Defensa tema de memoria}
\author{Jonathan Antognini C.\\
		Prof. Guía Dr. Luis Salinas\\
        Prof. Correferente M.Sc Paola Arce}
\institute[UTFSM]{Universidad Técnica Federico Santa María}
\date{\today}

\begin{document}
    \frame{\titlepage}
    \begin{frame}{\contentsname}
        \frametitle{Tabla de Contenidos} 
        \begin{multicols}{2}
        \tableofcontents
        \end{multicols}
    \end{frame}
	
    \section{Introducción}
        \subsection{Mercados Financieros}
            \begin{frame}
            \frametitle{Mercados Financieros} 
            \end{frame}
        \subsection{Serie de tiempo Forex}
            \begin{frame}
            \frametitle{Serie de tiempo Forex}
            \end{frame}
    \section{Descripción del problema}
        \subsection{Vector AutoRegressive}
            \begin{frame}
            \frametitle{Vector AutoRegressive} 
            \end{frame}
        \subsection{Vector Error Correction}
            \begin{frame}
            \frametitle{Vector Error Correction} 
            \end{frame}
        \subsection{Mínimos Cuadrados}
            \begin{frame}
            \frametitle{Mínimos Cuadrados}
            \end{frame}
	\section{GPU Computing}
		%\subsection{GPU Computing}
		%\subsection{NVIDIA CUDA}
	\section{Metodología}
        \subsection{Algoritmo Propuesto}
            \begin{frame}
            \frametitle{Algoritmo Propuesto}
            \end{frame}
        \subsection{Parámetros del Algoritmo}
            \begin{frame}
            \frametitle{Parámetros del Algoritmo}
            \end{frame}
        \subsection{Diagrama de clases}
            \begin{frame}
            \frametitle{Diagrama de clases}
            \end{frame}
    \section{Experimentos}
        \subsection{Selección de data}
            \begin{frame}
            \frametitle{Selección de data} 
            \end{frame}
        \subsection{Selección de Parámetros}
            \begin{frame}
            \frametitle{Selección de Parámetros}
            \end{frame}
        \subsection{Pruebas de Cointegración}
            \begin{frame}
            \frametitle{Pruebas de Cointegración}
            \end{frame}
        \subsection{Test de raíz unitaria}
            \begin{frame}
            \frametitle{Test de raíz unitaria}
            \end{frame}
        \subsection{Performance}
            \begin{frame}
            \frametitle{Performance}
            \end{frame}
        \subsection{Resultados}
            \begin{frame}
            \frametitle{Resultados}
            \begin{table}[ht!]
            \caption{Métricas del modelo, Frecuencia 60s}
            \label{tab:mapes_60s}
            \begin{center}
            \begin{adjustbox}{max width=\textwidth}
            \begin{tabular}{|l|l|c|c|c|c|c|c|c|c|}
            \hline
            \multicolumn{4}{|c|}{Model} & \multicolumn{3}{|c|}{In-sample} &
            \multicolumn{3}{|c|}{Out-of-sample} \\
            \hline
            \hline
            Método & L & P & e &
            MAPE & MAE& RMSE&
            MAPE & MAE& RMSE \\
            \hline
             OVECM  &   100  &  P=2& 0.0026  &  0.00263&  0.00085&  0.00114&  0.00309&  0.00094&  0.00131\\
             OVECM  &   400  &  P=5& 0.0041  &  0.00378&  0.00095&  0.00127&  0.00419&  0.00103&  0.00143\\
             OVECM  &   700  &  P=3& 0.0032  &  0.00323&  0.00099&  0.00130&  0.00322&  0.00097&  0.00132\\
             OVECM  &   1000 &  P=3& 0.0022  &
             \textbf{0.00175}&  \textbf{0.00062}&  \textbf{0.00087} &
             \textbf{0.00172}&  \textbf{0.00061}&  \textbf{0.00090}\\
            \hline
             SLVECM  &   100 &  P=2& -  &  0.00262&  0.00085&  0.00113&  0.00310&  0.00095&  0.00132\\
             SLVECM  &   400 &  P=5& -  &  0.00375&  0.00095&  0.00126&  0.00419&  0.00103&  0.00143\\
             SLVECM  &   700 &  P=3& -  &  0.00324&  0.00099&  0.00130&  0.00322&  0.00098&  0.00132\\
             SLVECM  &   1000 &  P=3& -  &
             \textbf{0.00174}&  \textbf{0.00061}&  \textbf{0.00087}&
             \textbf{0.00172}&  \textbf{0.00061}&  \textbf{0.00090}\\
            \hline
            \end{tabular}
            \end{adjustbox}
            \end{center}
            \end{table}

            \end{frame}
    \section{Conclusiones}
            \begin{frame}
            \frametitle{Conclusiones}
            \end{frame}
        \subsection{Trabajo Futuro}
            \begin{frame}
            \frametitle{Trabajo Futuro}
            \newpage
            Finalmente y como trabajo futuro, se propone:
            \begin{itemize}
             \item Probar distintos métodos para la selección de parámetros. 
             \item Buscar cointegración de distintas monedas. 
             \item Incluir el volumen asociado al precio.
             \item Probar resultados con alguna estrategia. 
             \item Implementar el arribo de datos desde un servidor de datos. 
            \end{itemize}

            \end{frame}
\end{document}
