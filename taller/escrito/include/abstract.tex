%\begin{spacing}{.97}
\vspace*{2cm}
\thispagestyle{empty}
%\begin{center}
{\bfseries \Huge Abstract }
\vspace{1.5cm}

\begin{quotation}
Lorem ipsum dolor sit amet, consectetur adipiscing elit. Aenean eget tellus
dignissim, porta leo id, placerat est. Nam a libero nec arcu hendrerit pharetra
at a mi. Donec sagittis quis est id tempor. Maecenas imperdiet, mi at
sollicitudin faucibus, tellus turpis auctor lorem, ac lacinia magna tortor sit
amet neque. Ut iaculis sit amet purus sed consequat. Nunc pretium est non erat
lacinia faucibus. Phasellus id rutrum mi. Morbi nunc nisi, mattis sit amet
purus eget, convallis tristique urna. Curabitur quis tellus placerat, eleifend
nisi eget, scelerisque quam. Nulla et sapien fermentum, dictum elit non,
elementum elit.

Morbi pretium dolor vel vestibulum pulvinar. Donec lobortis arcu malesuada
augue dictum, et pulvinar lectus viverra. Praesent rhoncus, sapien sed
porttitor volutpat, nisi ex mattis massa, tincidunt accumsan sapien sem et
elit. Maecenas sed sem accumsan, sollicitudin nunc quis, imperdiet lectus.
Maecenas ut arcu vitae odio suscipit sollicitudin. Pellentesque sed est lacus.
Quisque nec lorem sem. Nulla tristique tempus lacus, ornare maximus magna
interdum non. Proin aliquet quis ligula eget condimentum. Phasellus egestas in
libero posuere dapibus. Sed vestibulum ullamcorper arcu, eu tempus lacus
ultrices at. Ut id commodo erat, tincidunt suscipit mi. Donec suscipit lacus
sed ex molestie, nec dignissim nunc volutpat. Cras a mollis tortor, a egestas
enim. Vestibulum rhoncus orci eget purus volutpat, non viverra justo porttitor.
Donec eleifend efficitur tincidunt.

Ut a nulla tortor. Nam urna tortor, accumsan eu est a, rutrum ornare lectus.
Nam vestibulum libero sed consectetur vehicula. Sed non neque nec velit sodales
feugiat consequat eu tellus. Sed ut elit rutrum ipsum tristique tempus.
Phasellus sit amet porta dolor. Fusce sodales bibendum tellus vel ultrices.
Integer vitae laoreet eros.
%%
%% Motivation: Why do we care about the problem and the results?
%% ==============================================================
%%

%%
%% Problem statement: What problem are we trying to solve?
%% ========================================================
%%

%However, when one includes a particle whose mass dominates the
%potential, the numerical error accumulates and renders the integration of the
%system not reliable. This particle could be a massive black hole in a stellar
%system or a sun in a protoplanetary disc.
%%
%% Approach: How did we go about solving or making progress on the problem?
%% =========================================================================
%%

%In this work we present the first GPU algorithm to integrate semi-Keplerian systems
%that corrects the orbital evolution of the small bodies revolving around the most
%massive one.
%%
%% Results: What's the answer?
%% ===========================
%%

%%
%% Conclusions: What are the implications of our answer?
%% ======================================================
%%


{\bf Keywords:} keyword1, keyword2.
\end{quotation}

%\end{center}

\newpage
% Español

\vspace*{2cm}
\thispagestyle{empty}
%\begin{center}
{\bfseries \Huge Resumen }
\vspace{1.5cm}

\begin{quotation}
Lorem ipsum dolor sit amet, consectetur adipiscing elit. Aenean eget tellus
dignissim, porta leo id, placerat est. Nam a libero nec arcu hendrerit pharetra
at a mi. Donec sagittis quis est id tempor. Maecenas imperdiet, mi at
sollicitudin faucibus, tellus turpis auctor lorem, ac lacinia magna tortor sit
amet neque. Ut iaculis sit amet purus sed consequat. Nunc pretium est non erat
lacinia faucibus. Phasellus id rutrum mi. Morbi nunc nisi, mattis sit amet
purus eget, convallis tristique urna. Curabitur quis tellus placerat, eleifend
nisi eget, scelerisque quam. Nulla et sapien fermentum, dictum elit non,
elementum elit.

Morbi pretium dolor vel vestibulum pulvinar. Donec lobortis arcu malesuada
augue dictum, et pulvinar lectus viverra. Praesent rhoncus, sapien sed
porttitor volutpat, nisi ex mattis massa, tincidunt accumsan sapien sem et
elit. Maecenas sed sem accumsan, sollicitudin nunc quis, imperdiet lectus.
Maecenas ut arcu vitae odio suscipit sollicitudin. Pellentesque sed est lacus.
Quisque nec lorem sem. Nulla tristique tempus lacus, ornare maximus magna
interdum non. Proin aliquet quis ligula eget condimentum. Phasellus egestas in
libero posuere dapibus. Sed vestibulum ullamcorper arcu, eu tempus lacus
ultrices at. Ut id commodo erat, tincidunt suscipit mi. Donec suscipit lacus
sed ex molestie, nec dignissim nunc volutpat. Cras a mollis tortor, a egestas
enim. Vestibulum rhoncus orci eget purus volutpat, non viverra justo porttitor.
Donec eleifend efficitur tincidunt.

Ut a nulla tortor. Nam urna tortor, accumsan eu est a, rutrum ornare lectus.
Nam vestibulum libero sed consectetur vehicula. Sed non neque nec velit sodales
feugiat consequat eu tellus. Sed ut elit rutrum ipsum tristique tempus.
Phasellus sit amet porta dolor. Fusce sodales bibendum tellus vel ultrices.
Integer vitae laoreet eros.
%%
%% Motivation: Why do we care about the problem and the results?
%% ==============================================================
%%

%%
%% Problem statement: What problem are we trying to solve?
%% ========================================================
%%

%However, when one includes a particle whose mass dominates the
%potential, the numerical error accumulates and renders the integration of the
%system not reliable. This particle could be a massive black hole in a stellar
%system or a sun in a protoplanetary disc.
%%
%% Approach: How did we go about solving or making progress on the problem?
%% =========================================================================
%%

%In this work we present the first GPU algorithm to integrate semi-Keplerian systems
%that corrects the orbital evolution of the small bodies revolving around the most
%massive one.
%%
%% Results: What's the answer?
%% ===========================
%%

%%
%% Conclusions: What are the implications of our answer?
%% ======================================================
%%

{\bf Palabras claves:} keyword1, keyword2.

\end{quotation}
