%\begin{spacing}{.97}
\vspace*{2cm}
\thispagestyle{empty}
%\begin{center}
{\bfseries \Huge Abstract }
\vspace{1.5cm}

\begin{quotation}
The technical analysis of stock markets and foreign exchange market has been
used in order to forecast the behavior of the future prices of certain stocks
or currencies. Such analysis is based on the patterns detection of a finance
time series. This time series has a non-stationary and random walk behavior.

%El análisis técnico de los mercados bursátiles y de divisas se ha utilizado
%para intentar predecir el comportamiento futuro de los precios de un activo o
%una divisa. Dicho análsis se basa en la detección de patrones en una serie
%temporal financiera, las que son conocidas por su comportamiento no
%estacionario y comportamiento aleatorio. 

A interest pattern to address is the integration and cointegration between
the prices of certain stocks or currencies. When two or more time series are
cointegrated, even though both process are non-stationary, as in the case
of finance time series, there is a long term equilibrium relation which link
both series such that the relation achieves stationarity. Vector Error
Correction Model take advantage about this pattern.

%Un patrón de interés de estudio es la integración y cointegración entre los
%precios de ciertas divisas. Cuando dos o más series de tiempo están
%cointegradas, a pesar de que ambos procesos no sean estacionarios, como es el
%caso de las series de tiempo financieras, existe una relación de equilibrio a
%largo plazo que vincula ambas series tal que esta relación sea estacionaria. El
%modelo que permite aprovechar este partón es el Vector Error Correction Model
%(VECM).

The cointegration of pairs of currencies are modeled in several works, particularly
with FOREX data. Also through software tools such as Matlab, Eviews, etc. is possible
to forecast computationally the behavior of those time series.

%Existen varios trabajos donde modelan este problema con pares de monedas,
%particularmente con las series de tiempo del Foreign Exchange Market (FOREX).
%Además mediante herramientas computacionales como matlab, eviews, etc. es
%posible predecir computacionalmente el comportamiento de dichas series de
%tiempo.

However, an important aspect of this kind of series are their Online nature,
since every second more information is generated due the Ask or Bid orders.
This topic has not been addressed completely. VECM is a matricial model, which
of Online point of view, suffers update of part of the model. In order
to forecast is necessary solve a matricial system equation.

%Sin embargo un foco importante de este tipo de series es su caracter
%\emph{Online}, ya que a cada segundo se genera nueva información en el sistema
%debido a la compra o venta de divisas. Tema que no ha sido abordado
%completamente en la literatura. VECM es un modelo matricial, que desde el punto
%de vista \emph{Online}, sufre ciertas actualizaciones de valores, y para poder
%predecir es necesario resolver un sistema matricial.

In the present work VEC Model, Sliding Windows VEC Model and Online VEC Model
are presented. The last look forward to optimize computational calculations reducing
the number of times that the cointegration vectors are calculated. Class diagram
and their implementation associated to Use Cases was created, considering: data
reader, data re-sample in certain frequency, results graphics and metrics, etc.

%En este trabajo se presenta el modelo VEC, el modelo VEC para ventanas
%deslizantes y la implementación \emph{Online} que busca optimizar cálculos
%computacionales reduciendo la cantidad de veces que se calculan los vectores de
%cointegración. Para ello se creo un diagrama de clases asociado a los casos de
%uso del modelo, considerando entre ello: lectura de datos, resample a cierta
%frecuencia, gráficos de resultados, etc.

Test with different currencies was ran, ensuring the conditions to apply this
model. The results shows that the Online VEC Model version, reduce almost 50\%
the execution maintaining the accuracy of the algorithm. Also CUDA routines was
implemented in order to optimize the execution time.

%Se realizaron pruebas con diferentes monedas de FOREX asegurando las
%condiciones para poder aplicar el modelo. Los resultados muestran que la
%versión \emph{Online} del modelo VEC, disminuye cerca del 50\% el tiempo de
%ejecución sin comprometer la precisión del algoritmo. Además se implementaron
%rutinas en CUDA para optimizar los tiempos de ejecución.

{\bf Keywords:} VECM, VAR, FOREX, GPU.
\end{quotation}

\newpage
% Español

\vspace*{2cm}
\thispagestyle{empty}
%\begin{center}
{\bfseries \Huge Resumen }
\vspace{1.5cm}

\begin{quotation}
El análisis técnico de los mercados bursátiles y de divisas se ha utilizado
para intentar predecir el comportamiento futuro de los precios de un activo o
una divisa. Dicho análsis se basa en la detección de patrones en una serie
temporal financiera, las que son conocidas por su comportamiento no
estacionario y comportamiento aleatorio. 

Un patrón de interés de estudio es la integración y cointegración entre los
precios de ciertas divisas. Cuando dos o más series de tiempo están
cointegradas, a pesar de que ambos procesos no sean estacionarios, como es el
caso de las series de tiempo financieras, existe una relación de equilibrio a
largo plazo que vincula ambas series tal que esta relación sea estacionaria. El
modelo que permite aprovechar este partón es el Vector Error Correction Model
(VECM).

Existen varios trabajos donde modelan este problema con pares de monedas,
particularmente con las series de tiempo del Foreign Exchange Market (FOREX).
Además mediante herramientas computacionales como matlab, eviews, etc. es
posible predecir computacionalmente el comportamiento de dichas series de
tiempo.

Sin embargo un foco importante de este tipo de series es su caracter
\emph{Online}, ya que a cada segundo se genera nueva información en el sistema
debido a la compra o venta de divisas. Tema que no ha sido abordado
completamente en la literatura. VECM es un modelo matricial, que desde el punto
de vista \emph{Online}, sufre ciertas actualizaciones de valores, y para poder
predecir es necesario resolver un sistema matricial.

En este trabajo se presenta el modelo VEC, el modelo VEC para ventanas
deslizantes y la implementación \emph{Online} que busca optimizar cálculos
computacionales reduciendo la cantidad de veces que se calculan los vectores de
cointegración. Para ello se creo un diagrama de clases asociado a los casos de
uso del modelo, considerando entre ello: lectura de datos, resample a cierta
frecuencia, gráficos de resultados, etc.

Se realizaron pruebas con diferentes monedas de FOREX asegurando las
condiciones para poder aplicar el modelo. Los resultados muestran que la
versión \emph{Online} del modelo VEC, disminuye cerca del 50\% el tiempo de
ejecución sin comprometer la precisión del algoritmo. Además se implementaron
rutinas en CUDA para optimizar los tiempos de ejecución.

{\bf Palabras claves:} VECM, VAR, FOREX, GPU.

\end{quotation}
