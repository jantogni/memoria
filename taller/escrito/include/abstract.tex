%\begin{spacing}{.97}
\vspace*{2cm}
\thispagestyle{empty}
%\begin{center}
{\bfseries \Huge Abstract }
\vspace{1.5cm}

\begin{quotation}
The technical analysis of stock markets and foreign exchange market has been
used in order to forecast the behavior of the prices of a certain stock
or currency. Such analysis is based on the detection patterns of
a financial time series, which are known by its non-stationary and random
behavior.

%El análisis técnico de los mercados bursátiles y de divisas se ha utilizado
%para intentar predecir el comportamiento futuro de los precios de un activo o
%una divisa. Dicho análsis se basa en la detección de patrones en una serie
%temporal financiera, las que son conocidas por su comportamiento no
%estacionario y comportamiento aleatorio. 

A interest pattern to address is the integration and cointegration between the
prices of certain stocks or currencies, which establish that when two or more
time series are cointegrated, and considering that both processes are
non-stationary, (as in the case of financial time series), there is a long term
equilibrium relation which link both series such that the relation is
stationary.  The model that take advantage of this scenario is the Vector Error
Correction (VECM)

%Un patrón de interés de estudio es la integración y cointegración entre los
%precios de ciertas divisas. Cuando dos o más series de tiempo están
%cointegradas, a pesar de que ambos procesos no sean estacionarios, como es el
%caso de las series de tiempo financieras, existe una relación de equilibrio a
%largo plazo que vincula ambas series tal que esta relación sea estacionaria. El
%modelo que permite aprovechar este partón es el Vector Error Correction Model
%(VECM).

The cointegration of pairs of currencies are modeled in several works, particularly
with FOREX data, which through software tools such as Matlab, Eviews, etc.
it is possible to computationally forecast the behavior of those time series.

%Existen varios trabajos donde modelan este problema con pares de monedas,
%particularmente con las series de tiempo del Foreign Exchange Market (FOREX).
%Además mediante herramientas computacionales como matlab, eviews, etc. es
%posible predecir computacionalmente el comportamiento de dichas series de
%tiempo.

However, an important aspect of this kind of series are their Online nature,
since every second more information is generated due the Ask or Bid orders,
topic which had been not addressed completely.  VECM is a matricial model,
which from the Online point of view, suffers of values updates, which implies
the solution of a matricial system equation, to be able to forecast.

%Sin embargo un foco importante de este tipo de series es su caracter
%\emph{Online}, ya que a cada segundo se genera nueva información en el sistema
%debido a la compra o venta de divisas. Tema que no ha sido abordado
%completamente en la literatura. VECM es un modelo matricial, que desde el punto
%de vista \emph{Online}, sufre ciertas actualizaciones de valores, y para poder
%predecir es necesario resolver un sistema matricial.

On this work, we present the VEC, Sliding Windows VEC and Online VEC Models.
The objective of the Online VECM is to optimize the computational calculations,
reducing the number of times that the cointegration vectors are calculated.
Additionally, a class diagram and their implementation associated to Use Cases
was developed, considering: data reading, data re-sample in certain frequency,
graphics, metrics, etc.

%En este trabajo se presenta el modelo VEC, el modelo VEC para ventanas
%deslizantes y la implementación \emph{Online} que busca optimizar cálculos
%computacionales reduciendo la cantidad de veces que se calculan los vectores de
%cointegración. Para ello se creo un diagrama de clases asociado a los casos de
%uso del modelo, considerando entre ello: lectura de datos, resample a cierta
%frecuencia, gráficos de resultados, etc.

We present the result of a set of tests with different currencies, ensuring all
the conditions to apply this model. The results shows that the Online VEC Model
version, reduce almost 50\% the execution time, maintaining the algorithm
accuracy. Additionally, we developed a couple of CUDA routines, in order to
optimize the calculation.

%Se realizaron pruebas con diferentes monedas de FOREX asegurando las
%condiciones para poder aplicar el modelo. Los resultados muestran que la
%versión \emph{Online} del modelo VEC, disminuye cerca del 50\% el tiempo de
%ejecución sin comprometer la precisión del algoritmo. Además se implementaron
%rutinas en CUDA para optimizar los tiempos de ejecución.

{\bf Keywords:} VECM, VAR, FOREX, GPU.
\end{quotation}

\newpage
% Español

\vspace*{2cm}
\thispagestyle{empty}
%\begin{center}
{\bfseries \Huge Resumen }
\vspace{1.5cm}

\begin{quotation}
El análisis técnico de los mercados bursátiles y de divisas ha sido utilizado
para intentar predecir el comportamiento de los precios de un activo o una
divisa. Dicho análsis se basa en la detección de patrones en una serie temporal
financiera, las que son conocidas por su comportamiento no estacionario y
aleatorio. 


Un patrón de interés de estudio es la integración y cointegración entre los
precios de ciertas divisas, el cual establece que cuando dos o más series de
tiempo están cointegradas, y considerando que ambos procesos no sean
estacionarios (como es el caso de las series de tiempo financieras), existe una
relación de equilibrio a largo plazo, que vincula ambas series para que esta
relación sea estacionaria. El modelo que permite aprovechar este partón es el
Vector Error Correction Model (VECM).

Existen varios trabajos donde modelan este problema con pares de monedas,
particularmente con las series de tiempo del Foreign Exchange Market (FOREX),
las cuales mediante herramientas computacionales como Matlab, Eviews, entre
otros, es posible predecir computacionalmente el comportamiento de dichas
series de tiempo.

Sin embargo un foco importante de este tipo de series es su carácter
\emph{Online}, ya que a cada segundo se genera nueva información en el sistema
debido a la compra o venta de divisas, el cual no ha sido abordado
completamente en la literatura.  VECM es un modelo matricial, que desde el
punto de vista \emph{Online}, sufre ciertas actualizaciones de valores, lo que
implica que para poder realizar una predicción es necesario resolver un sistema
matricial.

En este trabajo se presenta el modelo VEC, el modelo VEC para ventanas
deslizantes y la implementación \emph{Online} que busca optimizar cálculos
computacionales, reduciendo la cantidad de cálculo de los vectores de
cointegración.  Complementariamente, se creó un diagrama de clases asociado a
los casos de uso del modelo, considerando entre ello: lectura de datos,
resampleo a cierta frecuencia, gráficos de resultados, etc.

Se realizaron pruebas con diferentes monedas de FOREX, asegurando las
condiciones necesarias para poder aplicar el modelo. Los resultados muestran
que la versión \emph{Online} del modelo VEC, disminuye cerca del 50\% el tiempo
de ejecución sin comprometer la precisión del algoritmo. Adicionalmente, se
implementaron rutinas utilizando CUDA para optimizar el cálculo.

{\bf Palabras claves:} VECM, VAR, FOREX, GPU.

\end{quotation}
