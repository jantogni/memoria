\parindent=1em

Es ampliamente conocido que la tecnología relacionada a la computación
de alto desempeño, se está haciendo presente en áreas de investigación
científica, de los cuales se desarrollan soluciones a problemas reales,
relacionados a la física, química, biología, informática, etc, como también
en el ámbito de los mercados financieros~\citep{watsham1997quantitative},
que es el área principal del presente documento.

El \emph{Algorithmic Trading} y el \emph{High Frequency Trading} son moneda
común en los mercados de capitales desarrollados de Estados Unidos y Europa,
llegando a representar más del 60\% del volumen operado en distintas clases de
activos. 
La tendencia indica que esta proporción de \emph{trading} automatizado seguirá
creciendo, dejando atrás al \emph{trading} manual, donde los programas
computacionales operaran entre si, removiendo al elemento humano de las
operaciones directas, relegándolo al elemento comercial y al mantenimiento de
estos sistemas y algoritmos. 
Dicho ésto, con el tiempo se ha cambiado la ocupación central del \emph{trader}
hacia personas con un conocimiento matemático y de programación,
que sean capaces del desarrollo de algoritmos eficientes para la compra y venta
de activos.

Esta memoria está enfocada a abordar un problema relacionado con las series
financieras de alta frecuencia y una forma particular de poder realizar
pronósticos tomando en cuenta las distintas características de naturaleza
propia de este tipo de datos. Se pretende abordar esta problemática con
metodologías computacionales, aplicando análisis matemáticos útiles para esta
área. El problema es de carácter financiero, por lo que es necesario
contextualizar el tema mediante conceptos, criterios y términos generales
asociados al área.

Se abordará el problema de predicción del valor de una cartera de activos
financieros con el Modelos de Vectores de Corrección del Error (VEC por sus
siglas en inglés). El principio detrás de este modelos es que existe una
relación de equilibrio a largo plazo entre variables económicas y que, sin
embargo, en el corto plazo puede haber desequilibrios. Con los modelos de
corrección del error, una proporción del desequilibrio de un período (el error,
interpretado como un alejamiento de la senda de equilibrio a largo plazo) es
corregido gradualmente a través de ajustes parciales en el corto plazo.  Los
parámetros de este modelo son obtenidos usando el método mínimos cuadrados.
Siendo este factor la motivación principal, ya que este involucra una serie de
cálculos, estimaciones y en sí, es computacionalmente costoso ya que
dependiendo de ciertos factores, como la cantidad de activos, se puede estar
trabajando con variables de gran tamaño (matrices). Puesto que en este tipo de
mercado la eficiencia es un factor clave, se pretende implementar el algoritmo
usando técnicas de High Performance Computing y Graphical Processing Unit.
\newpage
\noindent Los \emph{\textbf{objetivos principales}} de esta memoria son:
\begin{itemize}
 \item Implementar un modelo VEC y analizar su comportamiento con
series financieras de alta frecuencia.
 \item Optimizar dichos algoritmos usando computación de alto desempeño
(GPU) para mejorar su rendimiento.
\end{itemize} 

\noindent Los \emph{\textbf{objetivos secundarios}}:
\begin{itemize}
	\item Adaptar el modelo VEC para este tipo serie, encontrando
los parámetros apropiados para su funcionamiento.
	\item Evaluar la factibilidad de optimizar el modelo VEC mediante la
programación de alto desempeño (GPU).
\end{itemize}


\noindent \textbf{Organización de la Memoria}

\noindent Esta memoria se organizará con el siguiente esquema:
\begin{itemize}
 \item Capitulo 2: High Frequency Trading: En la primera sección se
explicarán las principales componentes de este tipo de mercado. En la
Segunda sección se definirá el problema a abordar.
 \item Capítulo 3: Antecedentes: Se presentan los fundamentos matemáticos y
modelos que se usarán durante el desarrollo de esta memoria.
 \item Capitulo 4: Graphics Processing Units: reseña general a HPC y el 
lenguaje de programación en CUDA.
 \item Capítulo 5: Metodología: se propone un algoritmo VEC y la metodología de
desarrollo. Cómo se seleccionarán los parámetros del modelo y un diagrama de
clase con las implementaciones realizadas.
 \item Capítulo 6: Experimentos: se presentará como se seleccionó la data y
parámetros del algoritmo. Se probará el algoritmo propuesto, y se compararán
los resultados entre las implementaciones realizadas.
 \item Capítulo 7: Conclusiones: se realizarán conclusiones generales del
trabajo realizado y se detallarán posibles trabajos futuros.
\end{itemize}
