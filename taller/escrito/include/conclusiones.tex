El presente trabajo de memoria, expone y valida con la ejecución de un conjunto
de pruebas, la aplicación de un  \emph{Online
Vector Error Correction Model} a series financieras de alta frecuencia.

Basándose en los resultados expuestos en el capítulo ~\ref{ch:experimentos}, se
puede observar que OVECM reduce considerablemente los tiempos de ejecución en
comparación al SLVECM sin comprometer la precisión de la solución. OVECM en
comparación con la versión SLVECM reduce el tiempo de ejecución debido
principalmente al ahorro de cómputo de los vectores de cointegración, los
cuales se calculan mediante el método de Johansen. 
La condición para obtener nuevos vectores de cointegración, es una métrica
(MAPE) de la muestra in-sample, la cual estima qué tan bien se ajusta el modelo
a la data real. Por otra parte, OVECM introduce dos funciones de optimización
de cálculo matricial para obtener el modelo de forma iterativa y no construir
el modelo completo en cada paso.  VECM, tal como otros algoritmos que deben
resolver sistemas matriciales, fue implementado con dos métodos: OLS y el
planteado por Coleman, también conocido como Ridge Regression. El segundo tiene
por objetivo evitar problemas que se puedan generar tanto numéricamente, como
también problemas propios de las matrices (rank-deficient). 
Sin embargo, la implementación RR para datos reales, no mejora los tiempos de
ejecución (si lo hace para datos de prueba). En cuanto a tiempo de ejecución,
el algoritmo toma mucho menos tiempo que la frecuencia con la que los datos
arriban, esto significa que el resultado puede ser usado en estrategias de
trading.

Por otra parte, en cuanto a implementación, se hizo un trabajo consistente para
futuras implementaciones, esto es, documentación y estructura de clases acorde
a las necesidades del algoritmo. Principalmente el código está hecho en Python,
lo cual genera una barrera de entrada baja al momento de jugar con el código.
Además se crearon Ipython Notebook con ejemplos de cómo usar cada clase,
métodos y algoritmos.

\newpage
\section{Trabajo Futuro}
Finalmente y como trabajo futuro, se propone:
\begin{itemize}
 \item Probar distintos métodos para la selección de parámetros. Para este
trabajo se utilizó Akaike Information Criterion, pero existen otros como
Schwarz Criterion, Hannan-Quinn Criterion, etc.
 \item Buscar cointegración de distintas monedas. En este trabajo se trabajó
únicamente con 4 monedas, por lo que sería interesante buscar otras monedas que
estén cointegradas. Además esto ayudaría a que el modelo tenga más variables
explicativas, con lo cual el modelo podría mejorar.
 \item Para este trabajo solo se utilizó el precio top del order book,
por lo que sería interesante también buscar una forma de incluir el volumen
asociado al precio.
 \item Probar resultados con alguna estrategia. Mediante el protocolo FIX
probar con datos reales, qué bien se ajusta el modelo (perder o ganar).
 \item Implementar el arribo de datos desde un servidor de datos. Actualmente
se baja data histórica de las monedas y se trabaja con ella, no hay conexión
directa con un servidor.
\end{itemize}

