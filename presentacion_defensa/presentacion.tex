\documentclass{beamer}
\usepackage{graphics}
\usepackage{url}
\usepackage{ulem}
\usepackage{beamerthemesplit}
\usepackage{hyperref}
\usepackage{wrapfig}
\usepackage[spanish,activeacute]{babel}
\usepackage[utf8]{inputenc}
\usepackage{listings}
\usepackage{color}
\usetheme{Warsaw}
\usepackage{pstricks}

%\pgfdeclareimage[height=1.3cm]{logo-izq}{img/alma.png}
%\pgfdeclareimage[height=1.3cm]{logo-der}{img/alma-off.png}
%\logo{\pgfuseimage{logo-der}}
%\setbeamertemplate{sidebar left}
%{
%\logo{\pgfuseimage{logo-izq}}
%\vfill %pone la imágen en la esquina inferior izquierda
%\rlap{\hskip0.1cm\insertlogo} %inserta la imágen
%\vskip15pt
%}


\title{High-Frequency Trading \\ \& \\ Graphics processing unit}
\subtitle{Defensa tema de memoria}
\author{Jonathan Antognini C.\\
		Luis Salinas C.}
\institute[]{Universidad Técnica Federico Santa María}
\date{\today}

\begin{document}
    \frame{\titlepage}
    \frame{\tableofcontents}
	\section{Introducción}
		\parindent=1.5em

Como se sabe, la tecnología se está haciendo presente en todas y cada una de
las áreas de investigación, como física, química, biología, computación, y lo
que es de interés para este documento es el área de los mercados financieros
~\citep{watsham1997quantitative}.

El Algorithmic Trading y el High Frequency Trading son moneda común en los
mercados de capitales desarrollados de Estados Unidos y Europa, llegando a
representar más del 60\% del volumen operado en distintas clases de activos. La
tendencia indica que esta proporción de trading automatizado seguirá creciendo,
dejando atrás al trading manual, donde los robots operaran entre si, removiendo
al elemento humano de las operaciones directas, relegándolo al elemento
comercial y al mantenimiento de estos sistemas y algoritmos, cambiando así la
posición central del trader hacia matemáticos y programadores que desarrollen
algoritmos eficaces y eficientes para la compra y venta de activos.

Esta memoria está enfocada a abordar un problema relacionado con las series
financieras de alta frecuencia y una forma particular de poder realizar
pronósticos tomando en cuenta las distintas características de naturaleza
propia de este tipo de datos. Se pretende abordar esta problemática con
metodologías computacionales, aplicando análisis matemáticos útiles para esta
área. El problema es de carácter financiero, por lo que es necesario
contextualizar el tema mediante conceptos, criterios y términos generales
asociados al área.

Se abordará el problema de predicción del valor de una cartera de activos
financieros con el Modelos de Vectores de Corrección del Error (VEC por sus
siglas en inglés). El principio detrás de este modelos es que existe una
relación de equilibrio a largo plazo entre variables económicas y que, sin
embargo, en el corto plazo puede haber desequilibrios. Con los modelos de
corrección del error, una proporción del desequilibrio de un período (el error,
interpretado como un alejamiento de la senda de equilibrio a largo plazo) es
corregido gradualmente a través de ajustes parciales en el corto plazo.  Los
parámetros de este modelo son obtenidos usando el método mínimos cuadrados.
Siendo este factor la motivación principal, ya que este involucra una serie de
cálculos, estimaciones y en sí, es computacionalmente costoso ya que
dependiendo de ciertos factores, como la cantidad de activos, se puede estar
trabajando con variables de gran tamaño (matrices). Puesto que en este tipo de
mercado la eficiencia es un factor clave, se pretende implementar el algoritmo
usando técnicas de High Performance Computing y Graphical Processing Unit.

Los \emph{\textbf{objetivos principales}} de esta memoria son:
\begin{itemize}
	\item Implementar un modelo VEC y analizar su comportamiento con
series financieras de alta frecuencia.
	\item Optimizar dichos algoritmos usando computación CUDA para aumentar su rendimiento.
\end{itemize} 

Los \emph{\textbf{objetivos secundarios}}:
\begin{itemize}
	\item Adaptar el modelo VEC para este tipo serie, encontrando
los parámetros apropiados para su funcionamiento.
	\item Evaluar la factibilidad de optimizar el modelo VEC mediante la
programación en CUDA.
\end{itemize}


\textbf{Organización de la Memoria}

Esta memoria se organizará con el siguiente esquema:
\begin{itemize}
 \item Capitulo 2: High Frequency Trading: En la primera sección se
explicarán las principales componentes de este tipo de mercado. En la
Segunda sección se definirá el problema a abordar.
 \item Capítulo 3: Antecedentes: Se presentan los fundamentos matemáticos y
modelos que se usarán durante el desarrollo de esta memoria.
 \item Capitulo 4: Graphics Processing Units: reseña general a HPC y el 
lenguaje de programación en CUDA.
 \item Capítulo 4: Metodología: se propone un algoritmo VEC y la metodología de
desarrollo. Cómo se seleccionarán los parámetros del modelo y un diagrama de
clase con las implementaciones realizadas.
 \item Capítulo 5: Experimentos: se presentará como se seleccionó la data y
parámetros del algoritmo. Se probará el algoritmo propuesto, y se compararán
los resultados entre las implementaciones realizadas.
 \item Capítulo 6: Conclusiones: se realizarán conclusiones generales del
trabajo realizado y se detallarán posibles trabajos futuros.
\end{itemize}

	\section{High-Frequency Trading}
		\frame{
	%\frametitle{High-Frequency Trading}
	\centerline{High-Frequency Trading}
}

	\section{Graphics processing unit}
		\section{Graphics Processing unit}
Descripción de gpu y sus ventajas.

\subsection{Nvidia Cuda}
Descripción y utilidad

	\section{Objetivos de la Memoria}
		\frame
{
\frametitle{Objetivos de la Memoria de Titulación}
}

	\section{Planificación de Trabajo}
		\frame
{
\frametitle{Programa de Trabajo}
}

	\section{Avances}
		\frame{
	\frametitle{Avances}
	\begin{itemize}
		\item Gestión de documentos y material.
		\item Análisis bibliográfico.
		\item Estudio inicial acerca de GPU y Series de tiempo.
		\item Test e implementaciones básicas en CUDA.
		\item Acceso al cluster del cti-hpc.
	\end{itemize}
}

	\section{Conclusiones}
		El presente trabajo de memoria, expone y valida con la ejecución de un conjunto
de pruebas, la aplicación de un  \emph{Online Vector Error Correction Model} a
series financieras de alta frecuencia.

Basándose en los resultados expuestos en el capítulo ~\ref{ch:experimentos}, se
puede observar que OVECM reduce considerablemente los tiempos de ejecución en
comparación al SLVECM sin comprometer la precisión de la solución. OVECM en
comparación con la versión SLVECM reduce el tiempo de ejecución debido
principalmente al ahorro de cómputo de los vectores de cointegración, los
cuales se calculan mediante el método de Johansen. 
La condición para obtener nuevos vectores de cointegración, es una métrica
(MAPE) de la muestra in-sample, la cual estima qué tan bien se ajusta el modelo
a la data real. Por otra parte, OVECM introduce dos funciones de optimización
de cálculo matricial para obtener el modelo de forma iterativa y no construir
el modelo completo en cada paso.  
VECM, tal como otros algoritmos que deben resolver sistemas matriciales, fue
implementado con dos métodos: OLS y el planteado por Coleman, también conocido
como Ridge Regression. El segundo tiene por objetivo evitar problemas que se
puedan generar tanto numéricamente, como también problemas propios de las
matrices (rank-deficient). El tiempo de ejecución de este método depende
directamente de la cantidad de filas, columnas y rank de la matriz, se observó
que en casos reales la implementación en GPU no tiene mejor performance que la
versión en CPU, sin embargo para matrices cuadradas de gran dimensión, la
versión GPU tiene mejor performance.
En cuanto a tiempo de ejecución general, el algoritmo toma mucho menos tiempo
que la frecuencia con la que los datos arriban, esto significa que el resultado
puede ser usado en estrategias de trading.

Por otra parte, en cuanto a implementación, se hizo un trabajo consistente para
futuras implementaciones, esto es, documentación y estructura de clases acorde
a las necesidades del algoritmo. Principalmente el código está hecho en Python,
lo cual genera una barrera de entrada baja al momento de jugar con el código.
Además se crearon Ipython Notebook con ejemplos de cómo usar cada clase,
métodos y algoritmos.

\newpage
\section{Trabajo Futuro}
Finalmente y como trabajo futuro, se propone:
\begin{itemize}
 \item Probar distintos métodos para la selección de parámetros. Para este
trabajo se utilizó Akaike Information Criterion, pero existen otros como
Schwarz Criterion, Hannan-Quinn Criterion, etc.
 \item Buscar cointegración de distintas monedas. En este trabajo se trabajó
únicamente con 4 monedas, por lo que sería interesante buscar otras monedas que
estén cointegradas. Además esto ayudaría a que el modelo tenga más variables
explicativas, con lo cual el modelo podría mejorar.
 \item Para este trabajo solo se utilizó el precio top del order book,
por lo que sería interesante también buscar una forma de incluir el volumen
asociado al precio.
 \item Probar resultados con alguna estrategia. Mediante el protocolo FIX
probar con datos reales, qué bien se ajusta el modelo (perder o ganar).
 \item Implementar el arribo de datos desde un servidor de datos. Actualmente
se baja data histórica de las monedas y se trabaja con ella, no hay conexión
directa con un servidor.
\end{itemize}


\end{document}
