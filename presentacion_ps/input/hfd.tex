\section{High Frequency Data}

\subsection{Definiciones}
\begin{frame}
\frametitle{Definiciones}

\begin{itemize}
	\item Tick: un tick corresponde a la unidad atómica de información tomada
directamente de un mercado financiero sobre un activo determinado. La
información contenida en un tick, tiene asociado un instante de tiempo, precio,
volúmen, y otros datos.
	\item High Frequency Data: se denomina a un conjunto de datos que contienen
reportes sobre la actividad de un mercado financiero. Esta información está
detallada en ticks, tomados en intervalos de tiempo variable, incluso en
fracciones de segundos.
\end{itemize}

Se define una serie temoral como una secuencia de observaciones $Y = {Y_t: t
\in T}$ medidas en ciertos instantes de tiempo y ordenados cronológicamente.
En particular, este tipo de serie captura observaciones de forma irregular en
el tiempo (serie no-homogénea).

\end{frame}

\subsection{Problema}
\begin{frame}
\frametitle{Característias}

El pronóstico de series de tiempo plantea la necesidad de conocer las
características más relevantes para tener mayor conocimiento y control de las
dinámicas que posee con el fin de poder determinar un estado futuro. En HFD,
algunas características son:

\begin{itemize}
	\item La frecuencia de los ticks depende del mercado y de su período
estacional. Puede llegar a fracciones de segundo.
	\item El tiempo entre ticks no es constante.
	\item La volatilidad, intensidad de operaciones y volúmenes transados
tienen forma de \emph{U}.
	\item Los cambios del precio diario están regidos a un conjunto finito de
valores, debido a reglas institucionales.
	\item En mercados de alta frecuencia se transmiten algunos ticks errados.
\end{itemize}}

\end{frame}

\begin{frame}
\frametitle{Problemas}
Este fenómeno del punto de vista estadístico, corresponde a un \emph{proceso
estocástico}, en el cual la información pasada juega un rol importante, y
además se debe considerar componentes aleatorios (no controlables).

Proceso estocástico: conjunto de variables aleatorias $X_t$ indexadas por un
índice $t$, dado que $t \in T$, con $T \subseteq R$. Donde $T$ puede ser
continuo si es un intervalo o discreto si es enumerable. Las variables
aleatorias $X_t$ toman valores en un conjunto que se llama \emph{espacio
probabilístico}.

Un espacio probabilístico es un espacio que se define en base a $(\Omega,
\mathcal B, P)$ en donde: $\Omega$ es el espacio muestral, $\mathcal B$ es el
álgebra de suceso, y $P$ es la función de probabilidad.

\end{frame}
