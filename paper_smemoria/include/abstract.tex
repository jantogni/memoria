En la vida cotidiana de la mayoría de los inversores individuales, se toman estrategias de compra y retención a largo plazo. Esto porque en su mayoría
no disponen tiempo o experiencia para este tipo de comercio. Sin embargo es conocido que esta área es muy lucrativa, y además existen profesionales en el área
conocidos como \emph{dealers}. En el presente las teconologías permiten a los individuos que no están trabajando para algun tipo de firma, tener a
mano este tipo de mercado. Pero esto no es novedad, este tipo de mercados ha encontrado una fuerte componente digital, con sistemas computacionales robustos,
que permiten a los usuarios realizar trading remotamente e incluso programando robots.
Los nuevos medios informáticos permiten trabajar con datos de alta frecuencia, como transacciones en intervalos de tiempo muy pequeño, lo cual se ha incluido
como una nueva e importante área de este tipo de mercado.
El objetivo de este documento es interiorizar al lector al respecto del \emph{electronic trading}, algunos enfoques del market making, al concepto de High-Frequency
Trading, y a la computación de alto desempeño.
