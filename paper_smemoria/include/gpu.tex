\section{GPU Computing}
%Descripción de gpu y sus ventajas.
Las siglas GPU provienen de Graphics Processing Unit, o en español Unidad de procesamiento gráfico. Para efectos de hardware, la GPU funciona como coprocesador,
pudiéndose utilizar de forma simultánea a la CPU y así aprovechar el potencial que puedan ofrecer ambas al mismo tiempo. Una GPU es un procesador
diseñado para llevar a cabo cálculos necesarios implicados en la generación de gráficos, ya sea, para un video juego, o como para una aplicación que utilice
gráficos en 2D, o 3D. Hoy en día las GPU son muy potentes y pueden incluso superar frecuencias de reloj de una CPU antigua.

Una GPU está formada por cientos de pequeños núcleos que trabajan juntos para procesar los datos de alguna aplicación. Esta arquitectura de procesamiento paralelo
masivo es la que proporciona al GPU su alta capacidad de cálculo. Existen numerosas aplicaciones aceleradas en la GPU que brindan una forma rápida de acceder
a la computación de alto desempeño (High Performance Computing).

El concepto implícito en todo este tema es el paralelismo, que es una forma de computación en la cual varios cálculos pueden realizarse simultáneamente,
siempre y cuando no existan dependencias secuenciales entre cálculos. Basándose en "divide y vencerás", principio que busca dividir los problemas grandes, para
obtener varios problemas pequeños, que son posteriormente solucionados en paralelo.

La evolución de las tarjetas gráficas ha venido acompañado de un gran crecimiento en el mundo de los videojuegos y las aplicaciones 3D, realizándose grandes
producciones de chips gráficos por parte de grandes fabricantes, como NVIDIA, AMD (ex ATI).

En los últimos años también han aparecido conjuntos de herramientas y compiladores que facilitan la programación de las GPUs, como por ejemplo, NVIDIA CUDA, que
cuenta con la comunidad más activa hasta la fecha en programación de GPUs.

\subsection{NVIDIA CUDA}
%Descripción y utilidad
Las primeras GPU fueron diseñadas como aceleradoras de gráficos y admitían apenas procesos específicos de funcionamiento fijo. En las últimas dos decadas, 
el hardware cada vez se volvió más programable, lo que culminó con la primera GPU de NVIDIA en los años 1999. En poco tiempo que se desarrollara el concepto de GPU,
investigadores empezaron a utilizar el rendimiento de estas tarjetas en cálculos con punto flotante.

Los dilemas que vivieron en futuro los programadores, fue que la programación para GPU estaba lejos de ser fácil, hasta que investigadores de la universidad de 
Stanford se propusieron reimaginar la GPU como un coprocesador de flujos.

Puesto que NVIDIA sabía que su hardware era bueno e iba creciendo muy rápido, debían combinarse con herramientas de harware y software intuitivas, invitaron
a un equipo de investigación y desarrollo, para empezar a evolucionar una solución que ejecturara el lenguaje de programación C a la perfección en el GPU.
Al reunión software y hardware, NVDIA lanzó al mercado CUDA en el año 2006. La competencia AMD tuvieron intentos forzosos en generar algo similar, pero su comunidad
de desarrollo no tuvo la misma motivación que si tuvo NVIDIA CUDA.

CUDA es una plataforma de computación paralela y un modelo de programación creado por NVIDIA.
La tecnología implementada por NVIDIA, es un entorno basado en el lenguaje C, que permite a los programadores escribir software para resolver problemas
computacionales complejos en menos tiempo aprovechando la gran capacidad de procesamiento paralelo de las GPU multinúcleo. Miles de programadores están utilizando
las herramientas gratuitas de desarrollo de CUDA (válidas para millones de GPU que circulan en el mercado), a fin de acelerar todo tipo de aplicaciones, desde
herramientas de codificación de audio y video, diseño de productos, investigación científica, etc.

Actualmente NVIDIA ofrece un kit de herramientas de CUDA, las cuales incluyen un compilador, bibliotecas de matemáticas, herramientas para corregir y optimizar
redimiento de aplicaciones. Encontrándose también con muestras de código, guías de programación, manuales de usuario, referencias de la interfaz de programación
de aplicaciones (API) y otra documentación para ayudar al usuario dar los primeros pasos en el área. Cabe destacar que ofrece todo esto de forma gratuita,
incluyendo NVIDIA Parallel Nsight for Visual Studio, el primer entorno de desarrollo del sector para aplicaciones masivas paralela que usan tanto GPU como CPU, esto
en sistemas operativos Windows.
