\section{Objetivos}
%Descripción de técnicas predictivas del spread usando series de tiempo.
Uno de los conceptos que ha incurrido actualmente en el área, es el High-Frequency Trading (HFT), en español corresponde a negociación de alta frecuencia.
Esto es un tipo de negociación que se lleva a cabo en los mercados financieros utilizando intensamente herramientas tecnológicas sofisticadas para obtener 
información del mercado y en función de la misma, intercambiar valores financieros como activos (vender o comprar). Las características principales:
\begin{itemize}
	\item Su componente cuantitativa involucra algoritmos informáticos para analizar datos del mercado e implementar estrategias de negociación
	\item Cada posición de inversión se mantiene solo durante breves períodos de tiempo, para rápidamente ejecturar la posición y comprar o vender, según
		convenga, y el activo de que se trate. En algunas ocaciones esto se realiza miles de veces en un solo día. Los períodos de tiempo en los que 
		se mantien las posiciones pueden ser de solo fracciones de segundos.
	\item Suelen llevarse a cabo por importantes salas de mercado, como fondos de inversión o bancos de inversión con carteras de activos
		de gran volumen y diversificadas.
	\item La característica clave y de interés, es que este tipo de neogicación es muy sencible a la velocidad de procesamiento del mercado y al propio 
		acceso al mercado.
\end{itemize}

Las oportunidades que se presentan en esta área son importantes, debido que los defensores del HFT destacan que la aplicación de la tecnología da liquidez al mercado,
abarata costos y contribuye a eliminar posibles ineficiencias en la formación de los precios. 

El HFT es una especialización de los Algorithmic Trading, que es la práctica de operar en los mercados financieros siguiendo unos algoritmos predefinidos. 
De este modo las operaciones se realizan por parte de máquinas conectadas a los mercados financieros en vez de por traders tecleando.
Actualmente existe una competición entre los High Frequency Traders para tener las máquinas más avanzadas y las conexiones más rápidas. Por supuesto esto se 
tiene que aplicar utilizando el software más rápido y avanzado del que sean capaces de diseñar. Es este motivo el cual hace nexo entre los mercados financieros 
y la computación de alto desempeño. 

Por lo tanto, lo que se pretende buscar es alguna aplicación de alguna estrategia de HFT bajo algoritmos hechos en HPC.
