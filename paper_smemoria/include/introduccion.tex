\section{Introducción}
Dentro de los estudios de la economía, se habla de las características de los mercados de competencia perfecta \cite{mankiw2011principles}:
\begin{itemize}
	\item Existe un elevado número de compradores y vendedores. La decisión individual de cada uno de ellos ejercerá escasa influencia sobre el mercado global.
	\item Homogeneidad de los productos, es decir, no existen diferencias entre productos que venden los oferentes.
	\item Transparencia del mercado. TOdos los participantes tienen pleno conocimiento de las condiciones generales en que opera el mercado.
	\item Libertad de entrada y salida de empresas. Todos los participantes, cuando lo deseen, podrán entrar o salir del mercado a costos nulos o casi nulos.
\end{itemize}.

Por otro lado la digitalización de los mercados mundiales es impulsada por varios fatores:
\begin{itemize}
	\item La globalización, expansión de redes de telecomunicaciones y con acceso fácil a la información.
	\item Nuevas tecnologías disponibles, más confiables, más seguras y más fáciles de usar.
	\item Los mercados y las empresas de corretaje necesitan crecer y captar nuevos clientes para no perder mercado frente a la competencia.
	\item Los clientes demandan mejor calidad de los servicios a costos más bajos.
	\item Hay una necesidad de contar con mercados más transparentes, más confiables, escalables, sin cuellos de botella y que permitan ser auditados fácilmente.
\end{itemize}

Al observar el ferviente crecimiento en las tecnologías de información y acceso a ello, se impulsó la creación de los mercados de trading electrónico. Los primeros
mercados de este tipo vieron la luz entre los años 1989 y 1992, pero el principal salto fue que en el año 1997 se dispusieron API's para los usuarios. Si se observa
como aumenta la tecnología y la accesibilidad a ella, era de imaginar los sucesos importantes que seguían en un futuro. Es así como el 2006 nacen las principales 
Electronic communications networks (concepto será explicado en la sección antecedentes generales): CBOT, NYMEX, NYSE. Ya en el 2007 con la web 2.0 y los smartphones 
del mercado, se crean sistemas de trading vía celulares y se perfilan como una de las mejores alternativas para los inversores individuales o minoristas.

Si bien es cierto, a cuando se le habla a cualquier persona respecto a las bolsas de comercios, lo primero que se le viene a la cabeza son una gran cantidad de 
personas comprando y vendiendo acciones, y llamando por teléfono de forma desenfrenada. Y en realidad los sistemas en la actualidad han ido dejando de funcionar así 
por algunas razones como: Falta de escalabilidad, dificultad para ampliar la base de clientes, insatisfacción del cliente, costos operativos, complejidad de 
integración, costo de desarrollar tecnología internamente, etc. Existen bastantes razones por las cuales este tipo de sistemas se vio destinado a desaparecer en el
tiempo, dejando la puerta abierta a los sistemas electrónicos.

Hasta ahora, se ha visto el enfoque de cómo se distribuye un sistema electrónico, de tal manera que se aproveche la tecnología, pero han quedado quizás muchas en
el aire, relacionadas al poder de una máquina para poder tomar decisiones, por ejemplo, ¿es reemplazable un \emph{dealer} por un computador o un sistema inteligente?.
Preguntas como estas y otras más se pretenden interiorizar conceptualmente en el desarrollo de este documento.
