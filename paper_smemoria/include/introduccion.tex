\section{Introducción}
A través de los años el hombre ha presentado un cambio radical en su nivel de vida; los conocimientos que él ha logrado acumular y aplicar han sido para su beneficio, 
y han cambiado radicalmente su modo de vivir. Existe una notable diferencia entre el hombre de hace unas cuantas décadas y el hombre moderno, tal diferencia se ha 
dado por el desarrollo de la ciencia, que está estrechamente relacionada con la innovacion tecnológica. Por esta razón se amplía el contenido de cómo ha 
evolucionado la ciencia y la tecnología en el mundo, su origen remoto, los países que más han aportado en esta área y su respectiva utilización, bien sea para 
el desarrollo o la destrucción.

Como se sabe, la tecnología se está haciendo presente en todas y cada una de las áreas de investigación, como física, química, biología, computación, y lo que es de 
interés para este documento es el área de los mercados financieros.

En tiempos pasados, las bolsas de valores estaban compuestas por agentes que compraban y vendían bonos o activos, esto de forma presencial, y dejando una impresión 
para una persona externa, de un cierto nivel de euforía y descontrol. Sin embargo la tecnología ha hecho una irrupción en esta área, permitiéndoles inicialmente
realizar las mismas acciones, pero desde plataformas computacionales online, lo cual, de un comienzo se pensó como: ahorrar espacio, recursos, facilidad de acceso, etc.
Pero por otro lado, se dejó abierta la posibilidad de incluir ciertas facciones robóticas dentro del actuar del ser humano, como caso sencillo, dejar un robot esperando
cierto precio de compra, y que el robot venda o compre. Cuando se habla de robot, se refiere a un programa que permita tomar decisiones y ejecutarlas también.

El primer gran salto desde el punto de vista de trading, fue el término asociado a High-Frequency Trading HFT, que es la negociación de acciones y productos derivados 
por medio de herramientas automatizadas, son programas informáticos desarrollados por las entidades financieras en estrecha colaboración entre el departamento de 
IT -técnico- y gestión de activos -funcional-.  La parte técnica entre otras muchas tares debe procesar, parametrizar y mapear grandes cantidades de datos 
relacionados con la cotización de los activos. En función de su configuración pueden enviar miles de órdenes -compra y venta- sobre un activo a lo largo de la sesión 
bursátil. Su esencia es puramente cuantitativa ya que los algoritmos, matrices de datos y parámetros son tratados mediante herramientas de estadística avanzada.  
Las órdenes en el HFT's se mantienen por un breve lapso de tiempo, la entrada y salida del mercado es muy rápida, normalmente por debajo del segundo. El principal 
motivo de la inmediatez se debe a que no son posiciones estratégicas en el capital de la sociedad cotizada.

En el siguiente documento se formularán conceptos básicos para que el lector pueda comprender el área en donde se está desarrollando el tema. Además se realizará
un nexo entre la computación de alto desempeño y este tipo de mercado.
