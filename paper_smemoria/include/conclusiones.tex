\section{Conclusiones}
Con la investigación realizada, se puede observar un amplio espectro de posibilidades de desarrollo. Sin embargo la memoria de pregrado se centrará en unir
los mercados de High-Frequency Trading (con datos reales) con GPU. La razón de querer realizar esto, es la capacidad que ofrecen estos dispositivos, sus ventajas
comparativas frente a cualquier programa que pueda ser procesado de forma secuencial en una CPU. A grandes rasgos la visión que se tiene para este problema es buscar
una cartera de acciones, con su secuencia histórica de valores (bid-ask), y el objetivo es intentar realizar una buena extrapolación para los siguientes precios
del activo en el mercado. Como se piensa trabajar con una 
cartera de acciones, lo más lógico sería procesar cada acción por separadas, y asociar un riesgo de compra venta por cada acción (dependiendo de factores de 
importancia). Como se trabajará en mercados de alta frecuencia de cambios de precios, una fracción de segundo es importantísima, por lo que se buscará implementar
una metodología eficaz y eficiente, ya sea en el modelado del problema, como también en la implementación posible.

El marco de esta memoria de pregrado incluye como profesor guía al PhD. Luis Salinas, y el centro de investigación de la universidad CTI-HPC, que gracias a sus nexos
y contactos en el área, conseguirán datos de prueba. Por otro lado se pretende utilizar el cluster que cuenta el centro de investigación y ahí realizar pruebas. \\ \\ \\ \\ \\
\begin{tabular}{|c|c|}
	\hline
	Actividad Realizado & Tiempo utilizado \\
	\hline
	Lectura documentos previos &  20 horas \\
	Preparación del documento (formato, diseño, etc) & 1 hora \\
	Desarrollo del documento$*$ & 14 horas \\
	Revisión Ortográfica y redacción & 2 hora \\
	\hline
\end{tabular}\\ \\ \\
$*$Al momento de leer cada documentos e realizaba un resumen pertienente que facilitaba la labor de redacción del documento.
